\section{Introduction}
%TODO Renan
% 1 page

Hydropower has an important share in the global electricity production, and
will continue to be a major source of renewable power-generation technology
\cite{iea}. Large hydropower projects have typically an average maintenance
cost of 2\% to 2.5\% of the investment cost per kW \cite{irena}, and a major
concern is the state of the runner's blades, which suffer cavitation and
abrasion phenomena, leading to water flow instability, excessive vibrations and
turbine efficiency reduction \cite{goldemberg2007energia}. Hard coating
techniques by thermal aspersion are used to reduce the erosion of the runner's
blade, greatly increasing its life cycle \cite{krella2011new}. 

The hard coating procedure requires a robotic system due to high precision,
speed, and the hazardous substances that are used, as propane and other gases.
Although sufficient for blade protection, the coating also has a life
cycle itself, thus it needs to be redone from time to time to ensure the
blade's protection from physical phenomena.

\begin{comment}
The studies also point out the potential increase
in efficiency and productivity with robot operators, besides of the improvement
in Health, Safety, and Environment (HSE) conditions, as robots can replace
humans in tasks performed in unhealthy, hazardous, and confined areas
\citep{pal}.

The use of robotics in O\&G industry represents great technological challenges
to overcome the following aspects of offshore environments \citep{chen}:

\begin{enumerate}[i)]
\item \emph{Atmospheric conditions} on offshore platforms are unfriendly, as
hydrocarbon resources can generate explosive and toxic gases;
\item \emph{Corrosive agents}: splashy salty water, salty air and corrosive
chemicals;
\item \emph{Weather} : high speed wind, rain, and hail. The relative humidity is
up to 100\% and ambient temperature can vary between $-30^{\circ}$C to
$50^{\circ}$C. Possibly highly radiant heat from equipment, and direct sunlight;
\item \emph{Constrained space}: complex structures for robots such as pipes,
flanges, tanks, and stairways.
\end{enumerate}

In the specific case of the Jirau hydroelectric dam, built on the Madeira
river, the number of suspended particles that the river carries intensifies the
abrasion phenomena, and Rijeza, a hard coating specialized company, identified cavitation erosion on blades, further reducing the coating life cycle.
Therefore, Jirau hydroelectric dam needs regular maintenance, which,
in the present situation, would require stoppage of the turbine, removing the
blades, positioning the blades for coating, coating application, turbine assembling, and recalibration. The downtime to perform all
maintenance can take up to two months, meaning a huge loss in power generation.

EMMA is an R\&D project by Fundação Coordena\-ção de Projetos, Pesquisas e
Estudos Tecnológicos (COPPETEC), in partnership with Rijeza company, Agência Nacional de
Energia Elétrica (ANEEL) and Energia Sustentável do Brasil (ESBR). Its first
stage is a technical feasibility study of a robotic system to perform
coating by thermal spray on hydraulic turbine blades within the turbine
environment. The project aims to significantly reduce the downtime for hard
coating process.
\end{comment}

\begin{comment}
The Oil \& Gas (O\&G) demand is expected to grow rapidly in the next decades
\citep{wna} and the need to obtain resources from hostile environments will
increase operation costs. Also, working conditions on offshore installations, such as
unfriendly atmosphere, heavy weather, extreme temperatures, and constrained
space are serious obstacles for O\&G companies. In order to be competitive, they
are looking into new technologies to be able to produce marginal fields. The
use of robotics in inspection, maintenance, and repair operations in O\&G
facilities could greatly improve efficiency, health and safety, while
decreasing operational and logistics costs.

In the specific case of Brazil, the O\&G industry is growing. The
recent discoveries of big oil fields in the pre-salt layer of the Brazilian
coast, located 300 km from the shore at depths of 5000-7000 m \citep{presal},
motivates the development of an offshore production system with high degree of
automation.

Recent studies forecast a substantial decrease in the level of human operation
and an increase in automation on future oil fields
\citep{skourup2009robotized}. The studies also point out the potential increase
in efficiency and productivity with robot operators, besides of the improvement
in Health, Safety, and Environment (HSE) conditions, as robots can replace
humans in tasks performed in unhealthy, hazardous, and confined areas
\citep{pal}.

The use of robotics in O\&G industry represents great technological challenges
to overcome the following aspects of offshore environments \citep{chen}:

\begin{enumerate}[i)]
\item \emph{Atmospheric conditions} on offshore platforms are unfriendly, as
hydrocarbon resources can generate explosive and toxic gases;
\item \emph{Corrosive agents}: splashy salty water, salty air and corrosive
chemicals;
\item \emph{Weather} : high speed wind, rain, and hail. The relative humidity is
up to 100\% and ambient temperature can vary between $-30^{\circ}$C to
$50^{\circ}$C. Possibly highly radiant heat from equipment, and direct sunlight;
\item \emph{Constrained space}: complex structures for robots such as pipes,
flanges, tanks, and stairways.
\end{enumerate}

Currently, the majority of the robotic systems in the O\&G industry are used for
subsea tasks, such as mapping of the seabed, and inspection and repair of
underwater equipment, risers and pipelines. However, recent research has focused
on robotic applications on the topside of oil platforms to perform inspection
and maintenance tasks, which include valve and lever manipulation, gas level
and leakage monitoring, acoustic anomalies diagnosis, and smoke and fire
detection.

Some examples found in the literature are:

The MIMROex inspection robot \citep{mimroex}, developed by the Fraunhofer
Institute of Manufacturing Engineering and Automation (IPA), is capable of
safely navigating in offshore environments, and autonomously executing
inspection tasks.

Sensabot \citep{sensabot}, a teleoperated inspection robot developed by
Carnegie Mellon University, was designed for severe weather and atmosphere,
being certified to operate in toxic, flammable and explosive environments. The
protoype was capable of safely operating on an onshore facility, executing all
its functionalities, including the level exchange through its cog rail
elevation system.

The SINTEF Topside Robotic System is an intelligent instrumentation system
designed to enable onshore operators to monitor and control the platform's
processes \citep{kyrkjebo2009robotic}.

In this paper, we present a general overview of the DORIS robot, and a detailed
description of the embedded electronics, power supply system and software architecture.

DORIS is an offshore inspection and monitoring robot being developed by
COPPE/UFRJ in collaboration with Petrobras and Statoil. The robot moves through
a rail carrying different sensors, processing sensor data \emph{in loco} or
storing it for future analysis. The sensors can identify abnormalities
such as intruders in restricted areas, abandoned objects, smoke, fire, and
liquid and gas leakages. The robot has an embedded manipulator, which enables
machinery vibration diagnosis, instruments reading, and sample taking
\citep{cba}.
\end{comment}

%%%%%%%%%%%%%%%%%%%%%%%%%%%%%%%%%%%%%%%%%%%%%%%%%%%%%
\begin{comment}
According to the world energy council, hydropower is the most flexible and
consistent of the renewable energy resources. Brazil is the second
country in hydropower production, and second with the highest
consumption of hydropower with a 70.000 MW installed capacity, and 433
hydroelectric plants in operation. Since Brazil is one of the world's richest
countries in water resources, and the hydropower is the most dominant across
the country, it motivates the development and investment in hydropower
generation.

In Brazil, the renovation and improvement of the built large power plants is
estimated to result in a potential increase of 32.000 MW
\citep{goldemberg2007energia}, a figure that can be achieved, in large part, by the maintenance of the
hydropower turbines. These turbines are constantly exposed to abrasion and
cavitation phenomena.

The cavitation phenomenon (Fig.~\ref{fig::cavitacao}) is detailed in \cite{escaler2006detection}, which outlines their types, occurrences and
effects in the different hydraulic turbines. This physical phenomenon can cause
erosions in the hydraulic turbines, leading to water flow instability,
excessive vibrations and turbine efficiency reduction.

\begin{figure}[h!]	
	\includegraphics[width=\columnwidth]{figs/intro/cavitacao2.png}
	\caption{Jirau hydraulic turbine's blade eroded by cavitation.}
	\label{fig::cavitacao}
\end{figure}

Hard coating techniques by thermal aspersion
are used to reduce the erosion of the turbine's blade from cavitation or
abrasion, thus increasing its life cycle. This solution is analogous to a paint
that protects walls from environment exposure. The hard coating procedure is performed
before the hydraulic turbine installation by a robotic manipulator. The
procedure requires a robotic system due to high precision, speed, and
the hazardous substances that are used, as propane and other gases.
Although sufficient for blade protection, the coating also has a life
cycle itself, thus it needs to be redone from time to time to ensure the
blade's protection from physical phenomena.

In the specific case of the Jirau hydroelectric dam, built on the Madeira
river, the number of suspended particles that the river carries intensifies the
abrasion phenomena, and Rijeza, a hard coating specialized company, identified cavitation erosion on blades, further reducing the coating life cycle.
Therefore, Jirau hydroelectric dam needs regular maintenance, which,
in the present situation, would require stoppage of the turbine, removing the
blades, positioning the blades for coating, coating application, turbine assembling, and recalibration. The downtime to perform all
maintenance can take up to two months, meaning a huge loss in power generation.

EMMA is an R\&D project by Fundação Coordena\-ção de Projetos, Pesquisas e
Estudos Tecnológicos (COPPETEC), in partnership with Rijeza company, Agência Nacional de
Energia Elétrica (ANEEL) and Energia Sustentável do Brasil (ESBR). Its first
stage is a technical feasibility study of a robotic system to perform
coating by thermal spray on hydraulic turbine blades within the turbine
environment. The project aims to significantly reduce the downtime for hard
coating process.

This document is divided as follows: section 2 describes, in detail, the
problem, contextualizes the reader in the Jirau environment and
describes the robot's tasks; section 3 surveys the state of the
art; section 4 describes the conceptual designs for the robot and mechanical
bases; finally, the section 5 concludes and outlines the next steps for the
EMMA project. 
\end{comment}


%%%%%%%%%%%%%%%%%%%%%%%%%%%%%%%%%%%%%%%%%%%%%%%%%
\begin{comment}
The importance of regular maintenance on hydraulic turbines have already been
established (ref EMMA-SOTA), given that its power output have a near 46\%
increase after maintenance. Providing a meaningful gain for highly dependent
countries like Brasil and Norway.

The efficiency of a hydraulic turbine is related at some extend to the hydraulic
profile of the runner blades and its degradation is mainly due to two
phenomena: cavitation and abrasion. As a protective measure, the blades are
hardcoated by a process called HVOF (High velocity oxy-fuel coating spraying),
which mitigate the damage caused by the aforementioned effects, but has to be
reapplied periodically.

The hardcoating process can take up to two months per turbine, including turbine
disassemble, blade's hardcoating on a specifically designed environment and
following remounting and calibration.

Aiming to reduce the downtime assossiated for the hardcoating process, robotic
\textit{in situ} solutions, i.e. inside the runner environemnt, consisting of a
insdustrial robotic arm mounted over a customized base are scrutinized. This is
the main objetive of the EMMA project, a R\&D project by Fundação Coordenação
de Projetos, Pesquisas e Estudos Tecnológicos (COPPETEC), in partnership with
Rijeza company, Agência Nacional de Energia Elétrica (ANEEL) and Energia
Sustentável do Brasil (ESBR).

Despite the generic approach of this article, the bulb turbine facilities are
different for each power plant. The reference hydroeletric power plant for the
ideas discussed here is the UHE Jirau, near Porto Velho(Brazil - RO). The
Madeira river, where the UHE Jirau is located, has a high concentration of
suspended particles which entails further abrasion on the blades, it also has a
low water level diference of 2 to 20 meter intensifing cavitation. For more
details on the UHE Jirau please reference (EMMA SOTA), however it is important
to recall that there are two main entry points for runner area: the top hatch
with a 35.7 cm diameter, just above the turbine, and the bottom hatch with a 80
cm diameter on the draft tube, 10 m away from the runner and with a 4m access
duct from exterior ground level. But, as a top hatch is not a standard across
other hydroeletric power plants, e.g. the UHE de Santo Antônio near UHE Jirau,
the solutions focus on the bottom hatch.

The workflow for the \textit{in situ} hardcoating can be thought as a
sequence of 4 minor jobs:
Enter the runner area with the robot; Move the robot and anchor it in some
suitable positions near the blade to be hardcoated; Calibrate the robot, in the
sense of identifying the relative positions of the robot, blade and the rest of
the environment; And finally, check if the the robot's arm can cover the whole
blade.

The following sections explore the developments on the EMMA project, they are
organized so to mimic the workflow:
section 2 exposes the ideas for the customized base and related logistics;
section 3 describes the calibration process which has to be done once the robot
is well placed, but before it's able to start coating; section 4 explore the
robot arm's limitations for performing the hardcoating; And lastly section 5
concludes and discusses the future steps for the EMMA project.
\end{comment}