\section{Introduction}
%RENAN - 1 pag

Hydropower has an important share in the global electricity production, and
will continue to be a major source of renewable power-generation technology
\cite{iea}. Large hydropower projects have typically an ave\-rage maintenance
cost of 2\% to 2.5\% of the investment cost per kW \cite{irena}, and a major
concern is the state of the runner's blades, which suffer cavitation and
abrasion phenomena. The erosion can lead to water flow instability, excessive
vibrations and turbine efficiency reduction \cite{goldemberg2007energia}, thus
hard coating techniques by thermal aspersion are used to greatly increase the
life cycle of runner's blades~\cite{krella2011new}.

The use of robotics for \textit{in situ} maintenance, and repair operations in
hydropower plants could greatly improve efficiency, health and safety, while
decreasing operational and logistics costs~\cite{hazel2012field}. The working
conditions on hydropower turbine installations are unfriendly, the atmospheric
conditions, high temperatures and humidity, and constrained space are
unfavorable to human operation. Also, some tasks, as the hard coating
procedure, requires a robotic system due to high precision, speed, and the
usage of hazardous substances (as propane and others).

In the specific case of Brazil, hydropower is the most important
power-generation technology. To support future economic growth, Brazil has
invested in additional hydroelectric facilities, for instance, the
14,000-megawatt Belo Monte dam along the Xingu River~\cite{eia}, and Jirau dam
along the Madeira river. At the latter, the number of suspended particles that
the river carries intensifies the abrasion phenomena, thus regular maintenance
is needed.

Currently, the majority of the robotic systems for \textit{in situ} hydropower
maintenance are used for repair tasks, such as inspection, welding, and
grinding. 

Some examples found in the literature are:

The Roboturb~\cite{roboturb} is a robotic manipulator to perform erosion
inspection, and welding on damaged runner's blades. It is composed of six
revolution joints and one prismatic joint coupled to a flexible rail, which may
be shaped and then fixed to the blade surface by a passive system of suction cups.

The Scompi~\cite{scompi} is a multipurpose manipulator, designed to
perform repairs on \textit{Francis} type turbines, as welding and grinding. It
is composed of five revolution joints (manipulator) and one prismatic joint
coupled to curved rails.

\textit{The Climber}~\cite{icm} is an inspection robot for wind and
hydroelectric turbines, to perform coating removal, surface cleaning and
coating application. It is a climbing robot with pneumatic adhesion and
locomotion by tracks.

In this paper, we present a general overview of the EMMA robot, and a detailed
description of the mechanics, robotic manipulator analysis, and calibration.

EMMA is a robotic system to perform \textit{in situ} hydropower runner's
blade hard coating, being developed by COPPE/UFRJ in collaboration with Agência
Nacional de Energia Elétrica (ANEEL) and Energia Sustentável do Brasil (ESBR). The system
is composed of an industrial manipulator that moves on a customized rail base, a
3D laser scanner for mapping, and sensors for positioning feedback. The
system will autonomously operate in a confined space, moves on a sloping
and slippery environment through a rail, identify the runner's blades, calibrate
its position, generates the path planning and performs the hard coating. The
project aims to significantly reduce the downtime for hard coating process.

\begin{comment}
This text is organized as follows: a general overview of the robot and its main
challenges are presented in Section \ref{sec:general_overview}, detailed
descriptions of the embedded electronics, the vehicle support system, power
supply system, and software architecture are taken in
Sections \ref{sec:electronics_overview}, \ref{sec:powersupply_overview}, and
\ref{sec:software} respectively.
In Section \ref{sec:results}, preliminary results are shown, and concluding
remarks are drawn in Section \ref{sec:conclusions}.
\end{comment}