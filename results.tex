\section{Results}

% 1 page

%TODO Gabriel: Results-Calibration
%TODO Estevão: Results-Mecanica
%TODO Renan: Results-Control

%._.-._.-._.-._.-._.-._.-._.-._.-._.- Calibration ._.-._.-._.-._.-._.-._.-._.-

%._.-._.-._.-._.-._.-._.-._.-._.-._.- Mechanics ._.-._.-._.-._.-._.-._.-._.-

The FEA analysis of the base verifies the Von Mises stress and displacements
along the structure slender members. The stress analysis determines the
integrity of the base due to the maximum loads of the robot. The displacements
determines if the structure provides a rigid base for the robot, since it is not
allowable to have big displacements, even in the elastic region of the material.
The maximum Von Mises stress was found for a particular case where the robot is
in the secondary rail in a distance of $800~mm$ from the rotational joint and
it is $5.78~MPa$, wich gives a Safety Factor of $34.6$. 
The displacement of the structure causes a maximum tranlation of $0.47~mm$ and a
resultant angular deflection of $0,0149^{\circ}$ in the coordinate system on
the base of the robot.

A last result from the FEA analysis is the Resultant Forces found in the end of
each anchor arm.
It is taken with the aim of properly size the magnetic bases capacity. So, if the
result in the main axis direction (\textit{x} in its local coordinate
system) of the arm is positive, the magnetic base is in compression to the
ground.
If the value is negative, the base is being tractioned, thus in this case, the magnetic capacity of the base needs to be
higher than this force to mantain the coupling. The
table~\ref{tab::forcas_ancor} shows the resultant forces in each direction,
where \textit{y} and \textit{z} are directions orthogonal to \textit{x}.

%._.-._.-._.-._.-._.-._.-._.-._.-._.- Control ._.-._.-._.-._.-._.-._.-._.-

