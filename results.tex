\section{Results}

% 1 page
%TODO Renan: Results-Intro
%TODO Gabriel: Results-Calibration
%TODO Estevão: Results-Mecanica
%TODO Renan: Results-Control

% Intro

Simulations and experimental tests were performed to verify the proposed
concepts. The following results were divided according to the EMMA system's
elements introduced in Sec.~\ref{solution}. 


\subsection{Robotic manipulator analysis}

As stated in Subsec.~\ref{manipulator}, simulations for the robotic
manipulator were implemented with OpenRave, and consist of the following steps:
blade's surface discretization; coating strategy; coating segmentation
(partitioning) and base positions computation; kinematics, dynamics and
manipulability.

The blade's surface discretization is an uniform sampling of the blade to
determine where to the coating directions should be. However, sampling the
actual geometric surface can lead to unwanted results due to possible
concavities. A simpler approach is to take the bounding box of the blade and
sample its surface uniformly (\textit{axis-aligned bounding box}). Once the
surface of the box is sampled, the intersection of the blade and a ray
originating from each point going inward is taken. The normal of the blade's
surface from each of these intersection points is taken to be the coating
direction. As an uniform sampling of the box does not mean an uniform sampling
of the blade, the box is oversampled and a 50~mm filter is applied to the
resulting sampling of the blade, by a multidimensional search key with k-d tree.
The result is an uniformely sampled blade, and the samples are spaced
50~mm from each other. These samples are translated 230~mm in respect with their
normal vectors, collision checks with the environment are made, and the
feasible samples are named \textit{coating samples}.%TODO Renan: FIGURA

The coating segmentation and base positions computation is to calculate the
required robotic manipulator's base positions to process all the \textit{coating
samples} with angle and distance tolerances. It is a brute force search, in
which the positions are uniformly sampled in the turbine's confined space (green dots
in Fig.). For each position, inverse kinematics (IKFast) are computed to
determine the robotic manipulator's joint parameters that provide the desired positions and
orientations of the end-effector. FIG X and Y show examples
of the algorithm, where black dots are coated points for the base positions in
green. At the end of the Alg.~\ref{alg:strategy}, the reachable samples are
in the Matrix~\ref{algvar:reachable}, which relates coating samples with base
positions, thus it is possible to create a coating strategy and to select the
simplest base positions. The result is the minimum required positions for the
robotic manipulator's base.
%TODO Renan: figura

\begin{algorithm}
\caption{Coating strategy}
\label{alg:strategy}
\begin{algorithmic}[1]
\ForAll{Base positions} 
		\ForAll{\textit{Coating samples}}
			\State Joints = IKFast(sample)
			\If{not Joints} 
				\State Joints = IKFast(sample,tol)
			\EndIf
			\If{Joints} 
				\State $\textrm{Reachable}[\textrm{pos}] += [\textrm{sample}]$
				\label{algvar:reachable}
			\EndIf
		\EndFor
\EndFor
\end{algorithmic}
\end{algorithm}

The kinematic approach described above is not enough to ensure that the robot
will reach the \textit{coating samples}. Maximum accelerations, decelerations,
torques, and jacobian singularities should be investigated and compared to the robotic
manipulator's specifications. The Newton-Euler method
\cite{sciavicco2000differential} was adopted for torque computation: $\tau =
M(q)\alpha + C(q,\omega)\omega + G(q)$, where $\tau$ is the joints' torques,
$M$ is the matrix of links' masses and moments of inertia, $\alpha$ is the
joints' accelerations, $q$ is the joints' angles, $\omega$ is the joints'
velocities, $C$ is the Coriolis matrix, and $G$ is the gravity vector.

In the dynamic approach, $M$ was estimated by the robotic manipulator's CAD
model. The angular accelerations are derived by differential kinematics:
$\alpha=J^+(a-\omega^TH\omega)$, where $H$ is the Hessian matrix
\cite{hourtash2005kinematic}, and $J$ is the Jacobian matrix. Therefore,
torques can be analytically estimated with the inverse dynamics in OpenRave.
Comparing estimated torques with the technical specifications, dynamics
simulations's results showed that the robotic manipulator should be placed at,
at least, 1400~mm distance from the blade's surface. Placing the robot nearer
would enhance the robotic manipulator's workspace, but it would increase the
torques also, above the technical specifications. Fig. shows joints' torques in
a color gradient way. %TODO Renan: FIGURA

\subsection{Base analysis}
%TODO Estevão: falar o programa utilizado para a análise, numero de elementos
% finitos e parametros necessarios para reproduçao da analise
The FEA analysis of the base verifies the Von Mises stress and displacements
along the structure's slender members. The stress analysis determines the
integrity of the base due to the maximum loads of the robotic manipulator. The
displacements determine if the structure provides a rigid base for the robotic
manipulator. According to the hard coating requirements, big displacements are
not even allowable in the elastic region of the material. %TODO Estevão: big
% displacements é generico. 

The maximum Von Mises stress was 5.78~MPa, which gives a factor of safety of
34.6. It was found for a particular case where the robotic manipulator is in
the secondary rail, 800~mm from the rotational joint. The displacement of the
structure causes a maximum translation of 0.47~mm and a angular deflection of
$0.0149^{\circ}$ in respect to robotic manipulator base's coordinate system.
%TODO Estevão: figura!

%TODO Estevão: Testes in situ com a base magnética

\begin{comment}
A last result from the FEA analysis is the Resultant Forces found in the end of
each anchor arm.
It is taken with the aim of properly size the magnetic bases capacity. So, if the
result in the main axis direction (\textit{x} in its local coordinate
system) of the arm is positive, the magnetic base is in compression to the
ground.
If the value is negative, the base is being tractioned, thus in this case, the magnetic capacity of the base needs to be
higher than this force to mantain the coupling. The
table~\ref{tab::forcas_ancor} shows the resultant forces in each direction,
where \textit{y} and \textit{z} are directions orthogonal to \textit{x}.

\begin{center}
\centering
\begin{tabular}{|c|c|c|c|}
\hline
\textbf{Braço}  & \textbf{Fx {[}N{]}} & \textbf{Fy {[}N{]}} & \textbf{Fz {[}N{]}} \\ \hline
\textbf{Anc\_1} & -1678               & -1966               & 39                  \\ \hline
\textbf{Anc\_2} & -3433               & -4076               & 40                  \\ \hline
\textbf{Anc\_3} & 131                 & 206                 & -569                \\ \hline
\textbf{Anc\_4} & -9556               & 6621                & 6037                \\ \hline
\textbf{Anc\_5} & -1566               & -1436               & 1461                \\ \hline
\textbf{Anc\_6} & 651                 & 160                 & -254                \\ \hline
\end{tabular}
\captionof{table}{Forças de reação em cada ponto de ancoragem}
\label{tab::forcas_ancor}
\end{center}
\end{comment}

\subsection{Calibration analysis}
%TODO Abelha: falar como funcionou o algoritmo, e testes do mapeamento in situ
