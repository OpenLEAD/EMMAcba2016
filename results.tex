\section{Results}

% 1 page
%TODO Renan: Results-Intro
%TODO Gabriel: Results-Calibration
%TODO Estevão: Results-Mecanica
%TODO Renan: Results-Control

% Intro

% Calibration

%TODO Estevão: falar o programa utilizado para a análise, numero de elementos
% finitos e parametros necessarios para reproduçao da analise
The FEA analysis of the base verifies the Von Mises stress and displacements
along the structure's slender members. The stress analysis determines the
integrity of the base due to the maximum loads of the robotic manipulator. The
displacements determine if the structure provides a rigid base for the robotic
manipulator. According to the hard coating requirements, big displacements are
not even allowable in the elastic region of the material. %TODO Estevão: big
% displacements é generico.

The maximum Von Mises stress was 5.78~MPa, which gives a factor of safety of
34.6. It was found for a particular case where the robotic manipulator is in
the secondary rail, 800~mm from the rotational joint. The displacement of the
structure causes a maximum translation of 0.47~mm and a angular deflection of
$0.0149^{\circ}$ in respect to robotic manipulator base's coordinate system.
%TODO Estevão: figura!

\begin{comment}
A last result from the FEA analysis is the Resultant Forces found in the end of
each anchor arm.
It is taken with the aim of properly size the magnetic bases capacity. So, if the
result in the main axis direction (\textit{x} in its local coordinate
system) of the arm is positive, the magnetic base is in compression to the
ground.
If the value is negative, the base is being tractioned, thus in this case, the magnetic capacity of the base needs to be
higher than this force to mantain the coupling. The
table~\ref{tab::forcas_ancor} shows the resultant forces in each direction,
where \textit{y} and \textit{z} are directions orthogonal to \textit{x}.

\begin{center}
\centering
\begin{tabular}{|c|c|c|c|}
\hline
\textbf{Braço}  & \textbf{Fx {[}N{]}} & \textbf{Fy {[}N{]}} & \textbf{Fz {[}N{]}} \\ \hline
\textbf{Anc\_1} & -1678               & -1966               & 39                  \\ \hline
\textbf{Anc\_2} & -3433               & -4076               & 40                  \\ \hline
\textbf{Anc\_3} & 131                 & 206                 & -569                \\ \hline
\textbf{Anc\_4} & -9556               & 6621                & 6037                \\ \hline
\textbf{Anc\_5} & -1566               & -1436               & 1461                \\ \hline
\textbf{Anc\_6} & 651                 & 160                 & -254                \\ \hline
\end{tabular}
\captionof{table}{Forças de reação em cada ponto de ancoragem}
\label{tab::forcas_ancor}
\end{center}
\end{comment}

% Kinematics

