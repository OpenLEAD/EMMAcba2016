\section{Results}

% 1 page
%TODO Renan: Results-Intro
%TODO Gabriel: Results-Calibration
%TODO Estevão: Results-Mecanica
%TODO Renan: Results-Control

% Intro

Simulations and experimental tests were performed to verify the proposed
concepts. The following results were divided according to the EMMA system's
elements introduced in Sec.~\ref{solution}. 


\subsection{Robotic manipulator analysis}

As stated in Subsec.~\ref{manipulator}, simulations for the robotic
manipulator were implemented with OpenRave, and consist of the following steps:
blade's surface discretization; coating strategy; coating segmentation
(partitioning) and base positions computation; kinematics and dynamics.

The blade's surface discretization uses the first part of the grasping algorithm
of OpenRave. It uniformly samples the object to determine where to
the coating directions should be. However, sampling the actual geometric surface
can lead to unwanted results due to possible concavities. A simpler approach is
to take the bounding box of the blade and sample its surface uniformly. Once
the surface of the box is sampled, the intersection of the blade and a ray
originating from each point going inward is taken. The normal of the blade's
surface from each of these intersection points is taken to be the coating
direction. %TODO Renan: FIGURA

\subsection{Base analysis}
%TODO Estevão: falar o programa utilizado para a análise, numero de elementos
% finitos e parametros necessarios para reproduçao da analise
The FEA analysis of the base verifies the Von Mises stress and displacements
along the structure's slender members. The stress analysis determines the
integrity of the base due to the maximum loads of the robotic manipulator. The
displacements determine if the structure provides a rigid base for the robotic
manipulator. According to the hard coating requirements, big displacements are
not even allowable in the elastic region of the material. %TODO Estevão: big
% displacements é generico. 

The maximum Von Mises stress was 5.78~MPa, which gives a factor of safety of
34.6. It was found for a particular case where the robotic manipulator is in
the secondary rail, 800~mm from the rotational joint. The displacement of the
structure causes a maximum translation of 0.47~mm and a angular deflection of
$0.0149^{\circ}$ in respect to robotic manipulator base's coordinate system.
%TODO Estevão: figura!

%TODO Estevão: Testes in situ com a base magnética

\begin{comment}
A last result from the FEA analysis is the Resultant Forces found in the end of
each anchor arm.
It is taken with the aim of properly size the magnetic bases capacity. So, if the
result in the main axis direction (\textit{x} in its local coordinate
system) of the arm is positive, the magnetic base is in compression to the
ground.
If the value is negative, the base is being tractioned, thus in this case, the magnetic capacity of the base needs to be
higher than this force to mantain the coupling. The
table~\ref{tab::forcas_ancor} shows the resultant forces in each direction,
where \textit{y} and \textit{z} are directions orthogonal to \textit{x}.

\begin{center}
\centering
\begin{tabular}{|c|c|c|c|}
\hline
\textbf{Braço}  & \textbf{Fx {[}N{]}} & \textbf{Fy {[}N{]}} & \textbf{Fz {[}N{]}} \\ \hline
\textbf{Anc\_1} & -1678               & -1966               & 39                  \\ \hline
\textbf{Anc\_2} & -3433               & -4076               & 40                  \\ \hline
\textbf{Anc\_3} & 131                 & 206                 & -569                \\ \hline
\textbf{Anc\_4} & -9556               & 6621                & 6037                \\ \hline
\textbf{Anc\_5} & -1566               & -1436               & 1461                \\ \hline
\textbf{Anc\_6} & 651                 & 160                 & -254                \\ \hline
\end{tabular}
\captionof{table}{Forças de reação em cada ponto de ancoragem}
\label{tab::forcas_ancor}
\end{center}
\end{comment}

\subsection{Calibration analysis}
%TODO Abelha: falar como funcionou o algoritmo, e testes do mapeamento in situ
