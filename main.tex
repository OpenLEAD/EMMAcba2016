% DO NOT EDIT THIS FILE
\documentclass[conference,harvard,brazil,english]{sbatex}
\usepackage[utf8]{inputenc} 
\usepackage{ae}
\usepackage{comment}
\usepackage{graphicx}
\usepackage{subfigure}


\begin{document} 

\title{A robotic system for \textit{in situ} hydropower turbine hard
coating} 

\author{Renan S. Freitas}{renan028@gmail.com} 
\address{Department of Electrical Engineering, COPPE
UFRJ, Rio de Janeiro, Brasil.}

\author[1]{Estevão F. Ferrão}{ef.ferrao@mecanica.coppe.ufrj.br} 
\author[1]{Gabriel Alcantara C. S.}{alcantara@poli.ufrj.br}
\author[1]{Eduardo Elael M. S.}{elael2@gmail.com}
\author[1]{Ramon R. Costa}{ramonrcosta@gmail.com}
   
\twocolumn[ 
\maketitle
\selectlanguage{english}
\begin{abstract}
	Hard coating of hydropower turbine's blades increase power generation
	efficiency and system's life cycle. Currently, blade coating is an \textit{ex situ}
	process limited to unassembled turbines. EMMA is a robotic
	solution for \textit{in situ} hard coating maintenance. The system operates
	in the confined turbine's environment, and complies the hard coating process
	requirements. The proposed system is composed of modular and customized rails,
	a robotic manipulator, and sensors for control, localization and mapping. The
	simulations and field tests validate the concepts considered so far and rise
	several challenges for future works.
\end{abstract}
\keywords{Hard coating, manipulator, robotics, robot
calibration, rail system, hydropower}
\selectlanguage{brazil}
\begin{abstract}
	O processo de metalização das pás de turbinas hidrelétricas aumenta a
	eficiência da geração de energia, e sua vida útil. Atualmente, o
	revestimento das pás é um processo realizado fora do ambiente da turbina. EMMA
	é uma solução robótica para manutenção \textit{in situ} por metalização
	de pás de turbinas hidrelétricas. O sistema opera no espaço confinado da
	turbina, e cumpre com as restrições do processo de metalização.
	O sistema é composto por trilhos modulares e customizados, um manipulador
	industrial robótico, e sensores para controle, mapeamento e localização. As simulações e testes de campo validam os
	conceitos considerados e levantam novos desafios para trabalhos
	futuros.
\end{abstract}
\keywords{Metalização, manipulador robótico, robótica,
calibração, sistemas em trilhos, hidrelétrica} ] 

\selectlanguage{english}
 
\section{Introduction}
%TODO Renan
% 1 page

Hydropower has an important share in the global electricity production, and
will continue to be a major source of renewable power-generation technology
\cite{iea}. Large hydropower projects have typically an average maintenance
cost of 2\% to 2.5\% of the investment cost per kW \cite{irena}, and a major
concern is the state of the runner's blades, which suffer cavitation and
abrasion phenomena. The erosion can lead to water flow instability, excessive
vibrations and turbine efficiency reduction \cite{goldemberg2007energia}, thus
hard coating techniques by thermal aspersion are used to greatly increase the
life cycle of runner's blades~\cite{krella2011new}.

The use of robotics for in situ maintenance, and repair operations in
hydropower plants could greatly improve efficiency, health and safety, while
decreasing operational and logistics costs~\cite{hazel2012field}. The working
conditions on hydropower turbine installations are unfriendly, the atmospheric
conditions, high temperatures and humidity, and constrained space are
unfavorable to human operation. Also, some tasks, as the hard coating
procedure, requires a robotic system due to high precision, speed, and the
usage of hazardous substances (as propane and others).

In the specific case of Brazil, hydropower is the most important
power-generation technology. To support future economic growth, Brazil has
invested in additional hydroelectric facilities, for instance, the
14,000-megawatt Belo Monte dam along the Xingu River~\cite{eia}, and Jirau dam
along the Madeira river. At the latter, the number of suspended particles that
the river carries intensifies the abrasion phenomena, thus regular maintenance
is needed.

Currently, the majority of the robotic systems for in situ hydropower
maintenance are used for repair tasks, such as inspection, welding, and
grinding. 

Some examples found in the literature are:

The Roboturb~\cite{roboturb} is a robotic manipulator to perform erosion
inspection, and welding on damaged runner's blades. It is composed of six
revolution joints and one prismatic joint coupled to a flexible rail, which may
be shaped and then fixed to the blade surface by a passive system of suction cups.

The Scompi~\cite{scompi} is a multipurpose manipulator, designed to
perform repairs on \textit{Francis} type turbines, as welding and grinding. It
is composed of five revolution joints (manipulator) and one prismatic joint
coupled to curved rails.

\textit{The Climber}~\cite{icm} is an inspection robot for wind and
hydroelectric turbines, to perform coating removal, surface cleaning and
coating application. It is a climbing robot with pneumatic adhesion and
locomotion by tracks.

In this paper, we present a general overview of the EMMA robot, and a detailed
description of the mechanics, manipulator analysis, and calibration.

EMMA is a robotic system to perform \textit{in situ} hydropower runner's
blade hard coating, being developed by COPPE/UFRJ in collaboration with Agência
Nacional de Energia Elétrica (ANEEL) and Energia Sustentável do Brasil (ESBR). The system
is composed of an industrial manipulator that moves on a customized rail base, a
3D laser scanner for mapping, and sensors for positioning feedback. The
system will autonomously operate in a confined space, moves on a sloping
and slippery environment through a rail, identify the runner's blades, calibrate
its position, generates the path planning and performs the hard coating. The
project aims to significantly reduce the downtime for hard coating process.

%This text is organized as follows: a general overview of the robot and its main
%challenges are presented in Section \ref{sec:general_overview}, detailed
%descriptions of the embedded electronics, the vehicle support system, power
%supply system, and software architecture are taken in
%Sections \ref{sec:electronics_overview}, \ref{sec:powersupply_overview}, and
%\ref{sec:software} respectively.
%In Section \ref{sec:results}, preliminary results are shown, and concluding
%remarks are drawn in Section \ref{sec:conclusions}.

\begin{comment}
require stoppage of the turbine, removing the
blades, positioning the blades for coating, coating application, turbine assembling, and recalibration. The downtime to perform all
maintenance can take up to two months, meaning a huge loss in power generation.
\end{comment}



%%%%%%%%%%%%%%%%%%%%%%%%%%%%%%%%%%%%%%%%%%%%%%%%%%%%%

\begin{comment}

The efficiency of a hydraulic turbine is related at some extend to the hydraulic
profile of the runner blades and its degradation is mainly due to two
phenomena: cavitation and abrasion. As a protective measure, the blades are
hardcoated by a process called HVOF (High velocity oxy-fuel coating spraying),
which mitigate the damage caused by the aforementioned effects, but has to be
reapplied periodically.

The hardcoating process can take up to two months per turbine, including turbine
disassemble, blade's hardcoating on a specifically designed environment and
following remounting and calibration.

Aiming to reduce the downtime assossiated for the hardcoating process, robotic
\textit{in situ} solutions, i.e. inside the runner environemnt, consisting of a
insdustrial robotic arm mounted over a customized base are scrutinized. This is
the main objetive of the EMMA project, a R\&D project by Fundação Coordenação
de Projetos, Pesquisas e Estudos Tecnológicos (COPPETEC), in partnership with
Rijeza company, Agência Nacional de Energia Elétrica (ANEEL) and Energia
Sustentável do Brasil (ESBR).


The workflow for the \textit{in situ} hardcoating can be thought as a
sequence of 4 minor jobs:
Enter the runner area with the robot; Move the robot and anchor it in some
suitable positions near the blade to be hardcoated; Calibrate the robot, in the
sense of identifying the relative positions of the robot, blade and the rest of
the environment; And finally, check if the the robot's arm can cover the whole
blade.

The following sections explore the developments on the EMMA project, they are
organized so to mimic the workflow:
section 2 exposes the ideas for the customized base and related logistics;
section 3 describes the calibration process which has to be done once the robot
is well placed, but before it's able to start coating; section 4 explore the
robot arm's limitations for performing the hardcoating; And lastly section 5
concludes and discusses the future steps for the EMMA project.
\end{comment}
\section{The problem}

%0.5 page

%TODO Renan: Problem

Hydropower runner's blades are typically eroded by cavitation and abrasion
phenomena, resulting in hydraulic profile deformation, thus efficiency
reduction. The High Velocity Oxygen Fuel (HVOF) coating process is a preventive
solution for erosion, and creates a lamellar structure, increasing
power generation efficiency. 

The runner's blade HVOF process consists of spraying coating particles by an
8~kg spray gun. To achieve the best coating layer, the spray gun should be at a
fixed 230~mm to 240~mm distance and $90^o \pm 30^o$ angle in respect
to the metallic surface plane of the blade; it should move with 40 m/s speed
along the blade; and a 3~mm coating step for full blade cover, due to the 5~mm
diameter flame.

\begin{comment}
In the case of the Jirau hydroelectric dam, the coating of turbine's blades is
performed before turbine assembling and installation. However, the abrasion due
to a large number of particles and sediment in the Madeira river and the recent
identified cavitation require recoating in short intervals \citep{santa2009slurry}.
\end{comment}
\section{Solution}

%2.5 pages
%TODO Renan: Solucao-Intro
%TODO Estevão: Solucao-Rails
%TODO Estevão: Solucao-Magnetic Base
%TODO Estevão: Solucao-Shutter
%TODO Renan: Solucao-Manipulator
%TODO Renan: Solucao-Software
%TODO Gabriel: Solucao-Software

Regarding the environment constraints detailed in Sec.~\ref{problem}, the
conceptual solution of EMMA is a mid-sized robotic manipulator with a modular
rail typed base. As the manipulator can not fully cover the blade surface in a
fixed position, and the robot locomotion is complex in the turbine's
environment, a customized rail should provide extra degrees of freedom (DOF) to
the system.

Overall simulations and analysis were performed using the Open Robotics
Automation Virtual Environment (OpenRave) for blade discretization, coating
strategy, coating segmentation (partitioning) and base positions, manipulator
kinematics and dynamics, and motion planning, in real-world
\cite{diankov2008openrave}. 

\begin{comment}
Several concepts for robotic base were
simulated and the simpler solution in constructive terms was chosen,
consisting of a Prismatic-Rotational-Prismatic-Prismatic (PRPP) base.
%TODO Estevâo: breve justificativa. ``Simpler'' ficou vago. Modular?



The coating solution used in industry today is to oversize the robot
in means of its reach, such that one fixed and large base is enough to do all 
the process without moving the robot's base. 
This is feasible because there are no constraints regarding to its size and no
other objects but the blade in an optimal position, in a specially designed room.

Because of the small access to the interior of the turbine and the narrow space
between the blades, this solution requires a midsize to small robot. 
These manipulators cannot reach all the surface of the blade without moving its
base, so an external base need to provide degrees of freedom (DOF) that allow
the manipulator to reach all the surface.
A geometric and kinematic study was made to find the required positions of the
robot's base that allow the complete coverage of the blade.
From this, were studied several base concepts regarding to the DOF and joints of
the base. The simpler solution in constructive terms was chosen, and consists of
a Prismatic-Rotational-Prismatic-Prismatic (PRPP) joints.
\end{comment}


\subsection{Robotic Manipulator}

\subsection{Base}
% 0.75 pages

The robot's base comprises two rails, forming two prismatic joints. The first,
or primary rail, is responsible for the robotic manipulator transportation
between the hatch and the blade, being parallel to the turbine axis. The
secondary rail is assembled from the first rail by a rotational joint for angle
adjustment, and it is parallel to the blade's surface. These rails guarantee the
full coating cover, as they allow the robotic manipulator movement through the
constrained space. A third prismatic joint is placed in the secondary rail for
height adjustment of the manipulator, as it can not reach all blade. 
%TODO Estevão melhorar, Renan revisou até aqui

The access hatch limits the size of the parts that can get
inside. The solution to this is to have a modular concept for the base, such
that it is possible to enter small parts or assemblies (modules) and mount
the base inside the turbine. It shall be easy to assemble and disassemble the
system manually, without the need of special tools. 

Each rail is composed of two parallel profiled rails with a four carriage setup.
In this configuration the reaction moments of the robot's base are cancelled by
a force couple provided by each couple of carriages and the loads are divided in
ore components.

A comercial modular profile system in aluminum is used to form the main
structure in wich the rails are mounted. 
The main advantages of this kind of system are a corrosion resistant and light
structure with a large flexibility in terms of geometry and modularity so that
it is possible to increase or decrease length, width or height of the structure
changing few parts, add anchor arms and many other accessories quickly.

Because of the constraints to access the turbine, the structure of the base
can't be oversized in terms of weight and geometry. So the result is a slender 
and ligth frame, such that a carefull analysis is necessary to evaluate the
structural integrity and rigidity due to the dynamic loads from the robot.
A Finite Element Analysis (FEA) is performed to study the stresses, strains and
resultant forces along the structure and are used to size the frame components,
like profile size, size of the anchors and its directions and to provide the
greater rigidity possible to the base. The rigidity is an important parameter
because of the precision required for the hardcoating process. If the base is
not rigid enough, it can propagate vibrations to the end-effector of the
arm with high amplitudes and compromise the quality of the coating.


\subsection{Magnetic bases}
%0.25 page

The interior of the turbine is curved and sloped. Besides that it is not
possible to modify its surface permanently, like drilling or welding. 
So it is a complicated task to fix the base of the system properly on the
ground.
However, almost all the surface of the turbine is a ferromagnetic steel
material, and it is possible to attach magnetic fixtures on it.
The comercials magnetic bases are non-permanent magnetic equipment used in
industry to hoist and transport materials in a plant. 
For this solution it is used to attach and anchor the structure of the base to
the ground. 
Tests were conducted in a sample of surfaces inside the turbine in order to
evaluate the load capacity of this equipment due to the different shapes,
surface finishing and curvatures that are found there.
The results concluded that this equipment is well suited for this application.


\subsection{Shutter}

The HVOF process requires a $40~m/min$ speed along the blade's surface, for a
best coating quality. 
The current industrial process uses an oversized robot that is able to
cover the blade from side to side. In this manner, the changes in direction, and
consequently changes in speed, occurs outside the blade's coating area.
This permits the robot to achieve the required speed again, when back to the
front of the surface.
While the HVOF gun is in the outside region of the blade, changing its
direction, it wastes the material to the environment or, most commonly to a
shadow plate.

The size of the robot for the \textit{in situ} solution makes this strategy
impossible, because the robot will be inevitably in an intern region of the blade and will
needs to change the direction of the HVOF gun several times in these
positions. 

Thus, the solution is basically to ''shut'' the process while the arm changes
its trajectory and open again when the required speed is achieved.
The shutter is a system that modifies the original circuit of the gases that
carries the coating particles to a new 2-way controlled circuit in wich there 
is a directional valve that changes the flow path from the thermal spray gun to
a return to tank circuit. 
The directional valve control module have informations about the pre-programmed
trajectory of the manipulator so that it is autonomously actuated.
In this way, there is much less waste of material, whereas the non-used
particles can be utilized again, and the hardcoating requirements can be met.


\subsection{Software}
\section{Results}

% 1 page
%TODO Renan: Results-Intro
%TODO Gabriel: Results-Calibration
%TODO Estevão: Results-Mecanica
%TODO Renan: Results-Control

% Intro

% Calibration

%TODO Estevão: falar o programa utilizado para a análise, numero de elementos
% finitos e parametros necessarios para reproduçao da analise
The FEA analysis of the base verifies the Von Mises stress and displacements
along the structure's slender members. The stress analysis determines the
integrity of the base due to the maximum loads of the robotic manipulator. The
displacements determine if the structure provides a rigid base for the robotic
manipulator. According to the hard coating requirements, big displacements are
not even allowable in the elastic region of the material. %TODO Estevão: big
% displacements é generico.

The maximum Von Mises stress was 5.78~MPa, which gives a factor of safety of
34.6. It was found for a particular case where the robotic manipulator is in
the secondary rail, 800~mm from the rotational joint. The displacement of the
structure causes a maximum translation of 0.47~mm and a angular deflection of
$0.0149^{\circ}$ in respect to robotic manipulator base's coordinate system.
%TODO Estevão: figura!

\begin{comment}
A last result from the FEA analysis is the Resultant Forces found in the end of
each anchor arm.
It is taken with the aim of properly size the magnetic bases capacity. So, if the
result in the main axis direction (\textit{x} in its local coordinate
system) of the arm is positive, the magnetic base is in compression to the
ground.
If the value is negative, the base is being tractioned, thus in this case, the magnetic capacity of the base needs to be
higher than this force to mantain the coupling. The
table~\ref{tab::forcas_ancor} shows the resultant forces in each direction,
where \textit{y} and \textit{z} are directions orthogonal to \textit{x}.

\begin{center}
\centering
\begin{tabular}{|c|c|c|c|}
\hline
\textbf{Braço}  & \textbf{Fx {[}N{]}} & \textbf{Fy {[}N{]}} & \textbf{Fz {[}N{]}} \\ \hline
\textbf{Anc\_1} & -1678               & -1966               & 39                  \\ \hline
\textbf{Anc\_2} & -3433               & -4076               & 40                  \\ \hline
\textbf{Anc\_3} & 131                 & 206                 & -569                \\ \hline
\textbf{Anc\_4} & -9556               & 6621                & 6037                \\ \hline
\textbf{Anc\_5} & -1566               & -1436               & 1461                \\ \hline
\textbf{Anc\_6} & 651                 & 160                 & -254                \\ \hline
\end{tabular}
\captionof{table}{Forças de reação em cada ponto de ancoragem}
\label{tab::forcas_ancor}
\end{center}
\end{comment}

% Kinematics


\section{Conclusion and future work}

% 0.5 pages

In this paper, we presented the concept solution and general overview of EMMA
project.

%TODO Elael: Conclusion and Future Work
\section*{Acknowledgements}
We gratefully acknowledge the financial support of Energia Sustentável do
Brasil and the ANEEL R\&D program (contract COPPETEC/UFRJ JIRAU 09/15
6631-0003/2015).


\bibliography{main} 
\end{document}