\section{Solution}

%2.5 pages
%TODO Estevão: Solucao-Intro
%TODO Estevão: Solucao-Rails
%TODO Estevão: Solucao-Magnetic Base
%TODO Estevão: Solucao-Shutter
%TODO Renan: Solucao-Manipulator
%TODO Renan: Solucao-Software
%TODO Gabriel: Solucao-Software

The coating solution used in industry today is to oversize the robot
in means of its reach, such that one fixed and large base is enough to do all 
the process without moving the robot's base. 
This is feasible because there are no constraints regarding to its size and no
other objects but the blade in an optimal position, in a specially designed room.

Because of the small access to the interior of the turbine and the narrow space
between the blades, this solution requires a midsize to small robot. 
These manipulator cannot reach all the surface of the blade without moving its
base, so an external base need to provide degrees of freedom (DOF) that allow
the manipulator to reach all the surface.
A geometric and kinematic study was made to find the required positions of the
robot's base that allow the complete coverage of the blade.
From this were studied several base concepts regarding to the DOF and joints of
the base. The simpler solution in constructive terms was chosen and consists of
a Prismatic-Rotational-Prismatic-Prismatic (PRPP) joints.


\subsection{Robotic Manipulator}

\subsection{Base}
% 0.75 pages

The robot's base comprises two rails, forming two prismatic joints. The first
rail, or primary rail, is responsible for the transportation between the turbine
access and the blade and it is parallel to the turbine axis. The secondary rail
is assembled from the first rail in a direction parallel to the transverse plane
of the blade's face. This rails allows the robot to move along the blade and to
cover a region of it at a time. A rotational joint is placed between the two
rails and is responsible for the ajdustment of the angle between them. 
Because of the limited reach of the manipulator, a last
prismatic joint in the height direction gives the robot the total reach required
to cover all regions of the blade.

The size of the turbine access limits the dimension of the parts that can get
inside. The solution to this is to have a modular concept for the base, such
that it is possible to enter small parts (modules) and assemble the base inside
the turbine. It shall be easy to mount and dismount the system manually, without
the need of special tools. A modular profile system in aluminum is used to form
the main structure in wich the rails are mounted. This kind of profile system
gives large flexibility in terms of geometry and modularity so that it is
possible to 




\subsection{Magnetic bases}

\subsection{Shutter}

\subsection{Software}