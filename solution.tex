\section{Solution}\label{solution}

%2.5 pages
%Renan: Solucao-Intro
%Renan: Solucao-Manipulator
%Estevão: Solucao-Rails (Renan revision)
%Estevão/Renan: Solucao-Shutter
%TODO Gabriel: Solucao-Calibration

The conceptual solution of EMMA is a mid-sized robotic manipulator with a
modular rail base. As the manipulator can not fully cover the blade's surface in
a fixed position, and the robot locomotion is complex in the turbine's
environment, a customized rail should provide extra degrees of freedom (DOF) to
the system. EMMA aims to be a generic solution to bulb turbines, being modular,
and versatile. It is an autonomous system for hard coating applications, thus
the following system's elements will be presented: the robotic manipulator; the
customized base; and the robot calibration.


\subsection{Robotic Manipulator}\label{manipulator}
Trying to propose an \textit{in situ} hard coating solution similar to
the \textit{ex situ} solution is a natural thought. Therefore, the first step of
EMMA's development was the research for industrial robotic manipulators. The
HVOF coating requirements and environment constraints demand a mid-sized
robotic manipulator, as seen in Sec.~\ref{problem} a large-sized one would not
be able to move inside the confined space, and a small-sized manipulator would not
have the required payload and speed. In EMMA's project, the
robotic manipulator will be responsible for the coating application, precision, and tool handling.

A market survey was conducted to determine the most suitable robotic
manipulator for the application. A 3D CAD model of the hydropower turbine was
built, and overall simu\-lations and analysis were performed using the Open
Robotics Automation Virtual Environment (OpenRave) \cite{diankov2008openrave}.
The simulations involve runner's blade discretization, coating strategy, base
positions computation for full blade cover, and robotic manipulator's kinematics
and dynamics. Among the analyzed mid-sized robotic manipulators, the Yaskawa Motoman MH12
robot was chosen due to its satisfactory workspace, and versatility.

The typical path of HVOF coating is a zigzagging trajectory,
i.e., it is a deceleration and acceleration process with direction changes. As
the \textit{ex situ} solution uses a large-sized robotic manipulator, the
end-effector's direction changes occur outside the blade's range, complying the
speed requirement in the blade's range. However, while deceleration or
acceleration, coating material is wasted to the environment or, most commonly,
to a shadow plate.

The mid-sized robot for \textit{in situ} operation makes this strategy
impossible, as, at some base positions, the robot will always be in the blade's
range. The proposed solution modifies the original circuit of the gases that
carry the coating particles to a new 2-way controlled circuit, in which there is
a directional valve that changes the flow path from the thermal spray gun to return to tank
circuit. The directional valve is autonomously actuated accordingly to the robotic
manipulator's trajectory, i.e., the valve will ``shut'' while end-effector
changes its direction. Therefore, the solution provides coating material
savings, as the non-used particles are stored for future usage, and the hard
coating speed requirement is met.

% \begin{figure}[h!]
%    \centering
%    \includegraphics[width=0.9\columnwidth]{figs/mecanica/Circuito_HVOF_mod_en.PNG}
%    \caption{Simplified view of HVOF circuit with shutter}
%    \label{fig::circuito_hvof}
% \end{figure}

Despite being able to coat approximately 50\% of the blade on a fixed
position, kinematic and dynamic analysis show that the MH12 robot cannot fully
cover it vertically or horizontally, demanding extra 2-DOF along the blade's
surface. The coating strategy reveals that the blade's top
extremities require an specific base position, reachable
with a primary rail between the blades. Therefore, the MH12 robot will require
at least four positions along the blade and one position on a primary
rail, between blades. 

\subsection{Base}
% 0.75 pages

EMMA's base comprises two rails, forming two prismatic joints (P). The
first, or primary rail, is parallel to the turbine axis, and it is responsible
for the transportation of the robotic manipulator from hatch to blade. The
secondary rail is parallel to the blade's surface, and it is assembled from the
first rail by a rotational joint (R). A third prismatic joint
is placed in the secondary rail for height adjustment of the robotic
manipulator. These rails guarantee the full coating, as they allow the
robotic manipulator movement in the confined space by a 4-DOF base, PRPP joints.

\begin{figure}
	\centering
	\includegraphics[width=.8\columnwidth]{figs/mecanica/EMMA_Base_Secundaria_01.PNG}
    \caption{Customized base: primary and secondary rails.}
    \label{fig:base}
\end{figure}

The hatch limits the size of the rail in terms of weight and geometry, thus a
modular concept was design, such that the small modular parts can be easily and
manually assembled inside the turbine. The base is a modular two parallel
profiled rail with a four carriage setup  \cite{SKF_2013}. In this
configuration, the reaction moments of the base are cancelled by a force couple provided by each couple of
carriages. The commercial modules are aluminum structures for corrosion resistance,
lightweight, geometric flexibility, and modularity, as it is possible to
increase/decrease the rail length/width/height by changing few parts or adding
anchor arms.

The resulting base is a slender and lightweight frame, such that a careful
analysis is necessary to evaluate the structural integrity and rigidity in
respect to the dynamic loads of the robotic manipulator. A Finite Element
Analysis (FEA) was performed to study the stresses, strains and resultant forces
along the structure. FEA is also used to size the frame components, such as
the profile's size, and to size the anchors, its directions, and attachment to
provide the greater base's rigidity. The rigidity is a major concern, it is
required for the hard coating process, since a non-rigid base propagates
vibrations to the robotic manipulator's end-effector with high amplitudes, which
may compromise coating quality.

The draft tube and runner area are conical shape structures, curved and
sloped. The environment modification, as wielding and drilling, should be
avoided, thus properly fixing the base on the ground is a major challenge in
EMMA project. The draft tube is composed of a ferromagnetic steel material,
hence magnetic fixtures may be suitable for base attachment. 

Common magnetic fixture products are temporary magnetic equipments used to hoist
and transport materials in an industrial plant. In EMMA, it is used to attach
and anchor the rail's modules to the ground.


\subsection{Calibration}

In opposition to \textit{ex situ} operations, in which the relative position
between the robot and the blade to be coated is fixed and \textit{a prior}
known, \textit{in situ} operations this assumptions cannot be made and it
is needed to aquire information from the environment in order to identify
and localize the elements of interest. In EMMA, the system calibration consists
in the indentification of the manipulator and blade, and their
pose estimation in respect to the turbine interior. Due the light conditions
inside the hydraulic circuit  a 3D laser scanner is used to gather
tridimensional information of the environment, including the robot manipulator
and base. The calibration process was divided in two different approach
depending on the element to be localized and its caracteristics. The possibility
to attach or install markers in known positions dictated the strategies to be
chosen, therefore the calibration is separeted in the pose estimation of the
robot and of the blade, as follows.
 
The attachment of markers on the robotic manipulator can be performed with high
precision and repeatability, thus reflective spheres were chosen as reference
points in the process of robot localization. These spheres are identified inside
the point cloud by a 3D Hough transform method \cite{camurri20143d}.

Alternatively, to install any marker on the
blade would require its own calibration to ensure a consistent reference point,
therefore the pose estimation of the blade must rely only on the intrinsec
properties of the its surface geometry. The information must be extracted from
the point cloud provided by the 3d laser sensor and compared to a reference
model previously stored. The characteristics of the point clouds are represented
by local descriptors, i.e., each point of interest on the blade is associated
with the information about its local neighborhood. Given the
sets of features,  it is possible to determine the correspondence between the
two sets. If enough correspondences are found in the scene, above a threshold,
the blade is identified and the position is determined \cite{Tombari2010a}. 

The computation of the descriptors is expensive and a subsampling of the point
clouds, both model and the scene 
