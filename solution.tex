\section{Solution}

%2.5 pages
%TODO Estevão: Solucao-Intro
%TODO Estevão: Solucao-Rails
%TODO Estevão: Solucao-Magnetic Base
%TODO Estevão: Solucao-Shutter
%TODO Renan: Solucao-Software
%TODO Gabriel: Solucao-Software

\begin{comment}
A solução de revestimento de uma pá de turbina \textit{in situ} requer um robô
de pequeno a médio porte, capaz de passar pelo limitado acesso da turbina. 
No entanto, a pá da turbina é uma peça com uma grande área a ser coberta e
nenhum manipulador comercial que atenda ao requisitos citados é capaz de
alcançar, de uma só posição, toda a sua extensão.
Assim, é necessário prover ao robô liberdade de posicionamento para realizar o
revestimento em pequenas regiões da pá, por posição de base.

Devido ao peso do manipulador e por questões de segurança, a sua movimentação no
interior da turbina não pode ser uma tarefa manual. Logo, uma base mecânica deve
ser capaz de levar o robô desde a escotilha até a posição ideal para o
revestimento, de forma segura e precisa. O dimensionamento desta base deve levar
em consideração todos os esforços de operação, como: o peso do sistema; as
cargas dinâmicas de movimentação do robô e o empuxo da pistola.

Foram estudados diversos conceitos para os graus de liberdade providos pela
base mecânica. O estudo destes conceitos estão detalhados no EMMA-DETAIL.
\end{comment}

The coating solution used in industry today is to oversize the robot
in means of its reach, such that one fixed and large base is enough to do all 
the process without moving the robot's base. 
This is feasible because there are no constraints regarding to its size and no
other objects but the blade in an optimal position, in a specially designed room.

Because of the small access to the interior of the turbine and the narrow space
between the blades, this solution requires a midsize to small robot. 
These manipulator cannot reach all the surface of the blade without moving its
base, so an external base need to provide degrees of freedom that allow the
manipulator to reach all the surface.
A geometric and kinematic study was made to find the required positions of the
robot's base that allow the complete coverage of the blade.
From this study 



\subsection{Robotic Manipulator}

\subsection{Rails}

\subsection{Magnetic bases}

\subsection{Shutter}

\subsection{Software}