\section{Solution}

%2.5 pages
%TODO Renan: Solucao-Intro
%TODO Estevão: Solucao-Rails
%TODO Estevão: Solucao-Magnetic Base
%TODO Estevão: Solucao-Shutter
%TODO Renan: Solucao-Manipulator
%TODO Renan: Solucao-Software
%TODO Gabriel: Solucao-Software

Regarding the environment constraints detailed in Sec.~\ref{problem}, the
conceptual solution of EMMA is a mid-sized robotic manipulator with a modular
rail typed base. As the manipulator can not fully cover the blade surface in a
fixed position, and the robot locomotion is complex in the turbine's
environment, a customized rail should provide extra degrees of freedom (DOF) to
the system.

Overall simulations and analysis were performed using the Open Robotics
Automation Virtual Environment (OpenRave) for blade discretization, coating
strategy, coating segmentation (partitioning) and base positions, manipulator
kinematics and dynamics, and motion planning, in real-world
\cite{diankov2008openrave}. 

\begin{comment}
Several concepts for robotic base were
simulated and the simpler solution in constructive terms was chosen,
consisting of a Prismatic-Rotational-Prismatic-Prismatic (PRPP) base.
%TODO Estevâo: breve justificativa. ``Simpler'' ficou vago. Modular?



The coating solution used in industry today is to oversize the robot
in means of its reach, such that one fixed and large base is enough to do all 
the process without moving the robot's base. 
This is feasible because there are no constraints regarding to its size and no
other objects but the blade in an optimal position, in a specially designed room.

Because of the small access to the interior of the turbine and the narrow space
between the blades, this solution requires a midsize to small robot. 
These manipulators cannot reach all the surface of the blade without moving its
base, so an external base need to provide degrees of freedom (DOF) that allow
the manipulator to reach all the surface.
A geometric and kinematic study was made to find the required positions of the
robot's base that allow the complete coverage of the blade.
From this, were studied several base concepts regarding to the DOF and joints of
the base. The simpler solution in constructive terms was chosen, and consists of
a Prismatic-Rotational-Prismatic-Prismatic (PRPP) joints.
\end{comment}


\subsection{Robotic Manipulator}

\subsection{Base}
% 0.75 pages

The robot's base comprises two rails, forming two prismatic joints. The first,
or primary rail, is responsible for the robotic manipulator transportation
between the hatch and the blade, being parallel to the turbine axis. The
secondary rail is assembled from the first rail by a rotational joint for angle
adjustment, and it is parallel to the blade's surface. These rails guarantee the
full coating cover, as they allow the robotic manipulator movement through the
constrained space. A third prismatic joint is placed in the secondary rail for
height adjustment of the manipulator, as it can not reach all blade. 
%TODO Estevão melhorar, Renan revisou até aqui

The access hatch limits the size of the parts that can get
inside. The solution to this is to have a modular concept for the base, such
that it is possible to enter small parts or assemblies (modules) and mount
the base inside the turbine. It shall be easy to assemble and disassemble the
system manually, without the need of special tools. 

Each rail is composed of two parallel profiled rails with a four carriage setup.
In this configuration the reaction moments of the robot's base are cancelled by
a force couple provided by each couple of carriages and the loads are divided in
ore components.

A comercial modular profile system in aluminum is used to form the main
structure in wich the rails are mounted. 
The main advantages of this kind of system are a corrosion resistant and light
structure with a large flexibility in terms of geometry and modularity so that
it is possible to increase or decrease length, width or height of the structure
changing few parts, add anchor arms and many other accessories quickly.

Because of the constraints to access the turbine, the structure of the base
can't be oversized in terms of weight and geometry. So the result is a slender 
and ligth frame, such that a carefull analysis is necessary to evaluate the
structural integrity and rigidity due to the dynamic loads from the robot.
A Finite Element Analysis (FEA) is performed to study the stresses, strains and
resultant forces along the structure and are used to size the frame components,
like profile size, size of the anchors and its directions and to provide the
greater rigidity possible to the base. The rigidity is an important parameter
because of the precision required for the hardcoating process. If the base is
not rigid enough, it can propagate vibrations to the end-effector of the
arm with high amplitudes and compromise the quality of the coating.


\subsection{Magnetic bases}
%0.25 page

The interior of the turbine is curved and sloped. Besides that it is not
possible to modify its surface permanently, like drilling or welding. 
So it is a complicated task to fix the base of the system properly on the
ground.
However, almost all the surface of the turbine is a ferromagnetic steel
material, and it is possible to attach magnetic fixtures on it.
The comercials magnetic bases are non-permanent magnetic equipment used in
industry to hoist and transport materials in a plant. 
For this solution it is used to attach and anchor the structure of the base to
the ground. 
Tests were conducted in a sample of surfaces inside the turbine in order to
evaluate the load capacity of this equipment due to the different shapes,
surface finishing and curvatures that are found there.
The results concluded that this equipment is well suited for this application.


\subsection{Shutter}

The HVOF process requires a $40~m/min$ speed along the blade's surface, for a
best coating quality. 
The current industrial process uses an oversized robot that is able to
cover the blade from side to side. In this manner, the changes in direction, and
consequently changes in speed, occurs outside the blade's coating area.
This permits the robot to achieve the required speed again, when back to the
front of the surface.
While the HVOF gun is in the outside region of the blade, changing its
direction, it wastes the material to the environment or, most commonly to a
shadow plate.

The size of the robot for the \textit{in situ} solution makes this strategy
impossible, because the robot will be inevitably in an intern region of the blade and will
needs to change the direction of the HVOF gun several times in these
positions. 

Thus, the solution is basically to ''shut'' the process while the arm changes
its trajectory and open again when the required speed is achieved.
The shutter is a system that modifies the original circuit of the gases that
carries the coating particles to a new 2-way controlled circuit in wich there 
is a directional valve that changes the flow path from the thermal spray gun to
a return to tank circuit. 
The directional valve control module have informations about the pre-programmed
trajectory of the manipulator so that it is autonomously actuated.
In this way, there is much less waste of material, whereas the non-used
particles can be utilized again, and the hardcoating requirements can be met.


\subsection{Software}