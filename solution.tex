\section{Solution}\label{solution}

%2.5 pages
%Renan: Solucao-Intro
%Renan: Solucao-Manipulator
%Estevão: Solucao-Rails (Renan revision)
%Estevão/Renan: Solucao-Shutter
%TODO Gabriel: Solucao-Calibration

EMMA robotic technology is described in this section. The following system's
elements will be presented: the robotic manipulator; the customized modular
base; and the robot calibration. 

\begin{comment}
The Jirau's t urbine is the case study of
EMMA, thus a 3D CAD model was built with
SolidWorks\raisebox{1ex}{\textregistered} for simulation and solution analysis.

Its hardware is composed
of a mid-sized robotic manipulator, a customized modular rail base, and
vision-based sensors. As the manipulator cannot fully cover the blade's surface in a fixed
position, and the robot locomotion is complex in the turbine's environment, a
customized rail is designed to provide extra degrees of freedom (DOF) to the
system. EMMA aims to be a generic solution to large bulb type turbines, being
modular, and versatile. 
\end{comment}


\subsection{Robotic Manipulator}\label{manipulator}
The HVOF coating requirements and environment constraints demand a mid-sized
robotic manipulator (Sec.~\ref{hvof}). The EMMA's manipulator will be responsible
for the coating application and accuracy.

A survey was conducted to determine the most adaptable off-the-shelf manipulator
for the application. Overall simu\-lations and analysis were
performed using the OpenRave \cite{diankov2008openrave}, an environment for
simulating motion planning algorithms for robotics. There are several tools for
dynamics simulation of robots: Gazebo, V-Rep, Webots, and others%
% \cite{ivaldi2014tools}
. OpenRave was selected because of its integral design for real-time control
and execution monitoring, the core functionality for kinematics operations and
physics simulations, and the ROS support, simplifying future software
integration.

The simulations were performed for manipulator's workspace analysis, and maximum
torque estimation. It involves runner's blade discretization, manipulator's base
position computations for full cover, and robotic manipulator's kinematics and dynamics. 

The blade's surface discretization is an uniform sampling to
determine where to the coating directions should be. The current approach is to
take the bounding box of the blade and sample its surface uniformly. Once the
surface of the box is sampled, the intersection of the blade and a ray
originating from each point going inward is taken. The normal of the blade's
surface from each of these intersection points is taken to be the coating
direction. As an uniform sampling of the box does not mean an uniform sampling
of the blade, the box is oversampled and a 50~mm filter is applied, by a
multidimensional search key with k-d tree. The samples are translated 230~mm in
respect with their normal vectors, and collision with the environment are checked. 

Manipulator's base position computations are to uniformly sample the turbine's
confined space and to calculate the required robotic manipulator's base
positions to process all the samples (blade discretization),  considering angle
and distance tolerances. It is a brute force search: for each position, inverse
kinematics are computed to determine the robotic manipulator's joint parameters that provide
the desired positions and orientations of the end-effector.

The kinematic approach described above is not enough to ensure that the robot
will reach the samples. Maximum accelerations, decelerations,
torques, and singularities should be investigated and compared to the
manipulator's specifications. To do this, we employ the well know relation:
$\tau = M(q)\ddot{q} + C(q,\dot{q})\dot{q} + G(q)$
\cite{sciavicco2000differential}, where $\tau$ is the joints' torques, $M$ is
the matrix of links' masses and moments of inertia, $q$ is the joints' angles,
$C$ is the Coriolis matrix, and $G$ is the gravity vector.

$M$ was estimated by the robotic manipulator's CAD model. The angular
accelerations are derived by differential kinematics:
$\alpha=J^+(a-\dot{q}^TH\dot{q})$, where $H$ is the Hessian matrix
\cite{hourtash2005kinematic}, $a=\ddot{X}$ is the linear accelerations, and $J$
is the Jacobian matrix. Therefore, torques can be analytically estimated with
the inverse dynamics in OpenRave.

% The typical path of HVOF coating is a zigzagging trajectory,
% i.e., it is a deceleration and acceleration process with direction changes. As
% the \textit{ex situ} solution uses a large-sized robotic manipulator, the
% end-effector's direction changes occur outside the blade's range, complying the
% speed requirement in the blade's range. However, while deceleration or
% acceleration, coating material is wasted to the environment or, most commonly,
% to a shadow plate.
% 
% The mid-sized robot for \textit{in situ} operation makes this strategy
% impossible, as, at some base positions, the robot will always be in the blade's
% range. The proposed solution modifies the original circuit of the gases that
% carry the coating particles to a new 2-way controlled circuit, in which there is
% a directional valve that changes the flow path from the thermal spray gun to return to tank
% circuit. The directional valve is autonomously actuated accordingly to the robotic
% manipulator's trajectory, i.e., the valve will ``shut'' while end-effector
% changes its direction. Therefore, the solution provides coating material
% savings, as the non-used particles are stored for future usage, and the hard
% coating speed requirement is met.

% \begin{figure}[h!]
%    \centering
%    \includegraphics[width=0.9\columnwidth]{figs/mecanica/Circuito_HVOF_mod_en.PNG}
%    \caption{Simplified view of HVOF circuit with shutter}
%    \label{fig::circuito_hvof}
% \end{figure}

 

\subsection{Base}
% 0.75 pages

EMMA's base comprises two rails, forming two prismatic joints (P). The
first, or primary rail, is parallel to the turbine axis, and it is responsible
for the transportation of the robotic manipulator from hatch to blade. The
secondary rail is parallel to the blade's surface, and it is assembled from the
first rail by a rotational joint (R). 

\begin{figure}
	\centering
	\includegraphics[width=.9\columnwidth]{figs/mecanica/EMMA_Base_Secundaria_02.PNG}
    \caption{Customized base: primary and secondary rails.}
    \label{fig:base}
\end{figure}

The hatch and turbine' space can limit the size of the rail in terms of weight
and geometry, thus a modular concept was design, such that the small modular parts can be easily and
manually assembled inside the turbine. The base is a modular two parallel
profiled rail with a four carriage setup%  \cite{SKF_2013}
. In this
configuration, the reaction moments of the base are cancelled by a force couple provided by each couple of
carriages. The commercial modules are aluminum structures for corrosion resistance,
lightweight, geometric flexibility, and modularity, as it is possible to
increase/decrease the rail length/width/height by changing few parts or adding
anchor arms.

The resulting base is a slender and lightweight frame, such that a careful
analysis is necessary to evaluate the structural integrity and rigidity in
respect to the dynamic loads of the robotic manipulator. A Finite Element
Analysis (FEA) was performed to study the stresses, strains and resultant forces
along the structure. FEA is also used to size the frame components, such as
the profile's size, and to size the anchors, its directions, and attachment to
provide the greater base's rigidity. The rigidity is a major concern, it is
required for the hard coating process, since a non-rigid base propagates
vibrations to the robotic manipulator's end-effector with high amplitudes, which
may compromise coating quality.

The draft tube and runner area are conical shape structures, curved and
sloped. The environment modification, as wielding and drilling, should be
avoided, thus properly fixing the base on the ground is a major challenge in
EMMA project. The draft tube is composed of a ferromagnetic steel material,
hence magnetic fixtures is suitable for base attachment. 

\begin{comment}
Common magnetic fixture products are temporary magnetic equipments used to hoist
and transport materials in an industrial plant. In EMMA, it is used to attach
and anchor the rail's modules to the ground.
\end{comment}

\subsection{Calibration}
 
In hard coating \textit{in situ} operations, the relative position between the
manipulator and the blade is not fixed. The system calibration consists in the
indentification of the manipulator and blade, and their pose estimation in
respect to the turbine interior. Due to the light conditions inside the
hydraulic circuit a 3D laser scanner is used to gather 
information of the environment, including the  manipulator and base.

\begin{comment}
The calibration process was divided in two different approach depending on the
element to be localized and its caracteristics. The possibility to attach or
install markers in known positions dictated the strategies to be chosen,
therefore the calibration is separeted in the pose estimation of the robot and
of the blade, as follows.
\end{comment}
   
The attachment of markers on the robotic manipulator can be performed with high
precision and repeatability, thus reflective spheres were chosen as reference
points in the process of robot localization. These spheres are identified inside
the point cloud by the 3D Hough Transform method \cite{camurri20143d}. The 3D
Hough Transform, as the 2D Hough Transform, is the search of an object on the
(discretized) parametric space. A sphere has four parameters: three for the
position of its center, and one for its radius. For each point on the point
cloud, it is assigned a collection of voxels on the discretized parametric
space, corresponding to the possible spheres passing through that point. As in a
voting process, the voxels with the greatest number of points assigned to define
the parameters of the most probable spheres. There might be computational issues
depending on the size of the parameter space, but it can be mitigated by
exploiting previous knowledge regarding the expected radius and viable region
for the robot inside the enviroment. Further improvements can be achieved
through the use of the normal vector at each point, exploiting the fact that
the center of the sphere is in the direction of the normal, thus reducing the
number of assigned voxels per point and in consequence the computational effort.

Fixing any marker on the blade would require its own calibration to ensure a
consistent reference point, thus the pose estimation of the blade must rely only
on the intrinsec properties of its surface geometry. The information must be extracted from
the point cloud (the scene) provided by the 3D laser sensor and
compared to a reference model previously stored. The characteristics of the point clouds are represented
by local descriptors, i.e., each interest point on the blade is associated
with a piece of information about its local neighborhood. Given the
sets of features from the model and the scence,  it is possible to determine the
correspondence between their descriptors. If enough correspondences are found
in the scene, above a threshold, the blade is identified and the position can be
determined \cite{Tombari2010a}.

As the blade is a large and smooth surface, the neighborhood of each point may
introduce similiar information, creating ambiguous descriptors that degrade the
matching perfomance, thus it is fundamental to wisely pick the interest points.
The ambiguous descriptors provide information about perpendicular translation to
the supporting plane, and the two rotations associated with it, thus no
information about the other DOFs, making the alignment to ``slide'' from the
correct transformation. Therefore, the interest points were sampled to
diversify the normal vectors' directions \cite{Rusinkiewicz2001}, allowing to
reduction of the number of samples to have a descriptor associated, when
compared to a uniform sampling, and maintaning the computational cost low.


Once the descriptors were estimated, the correspondences are determined if the
euclidean distance between a descriptor in the scene and in the model is lower
than a threshold. Each correspondence vote for a specific pose and scale factor in the
Hough space. After an instance of the model is found, it is performed the 
Iterative Closest Point (ICP) matching with the full resolution point clouds to
realize a fine alignment and compensate any discrepancy introduced by the
sampling. With the position of the blade and the manipulator in respect to a
common coordinate system, i.e., the origin of the laser sensor, it is possible
to determine the transformation between them and this information can be fed to
the trajectory and coverage algorithms.



