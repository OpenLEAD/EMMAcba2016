\section{Solution}\label{solution}

%2.5 pages
%Renan: Solucao-Intro
%Renan: Solucao-Manipulator
%Estevão: Solucao-Rails (Renan revision)
%Estevão/Renan: Solucao-Shutter
%TODO Gabriel: Solucao-Calibration

EMMA robotic technology is described in this section. The following system's
elements will be presented: the robotic manipulator; the customized modular
base; and the robot calibration. 

\begin{comment}
The Jirau's t urbine is the case study of
EMMA, thus a 3D CAD model was built with
SolidWorks\raisebox{1ex}{\textregistered} for simulation and solution analysis.

Its hardware is composed
of a mid-sized robotic manipulator, a customized modular rail base, and
vision-based sensors. As the manipulator cannot fully cover the blade's surface in a fixed
position, and the robot locomotion is complex in the turbine's environment, a
customized rail is designed to provide extra degrees of freedom (DOF) to the
system. EMMA aims to be a generic solution to large bulb type turbines, being
modular, and versatile. 
\end{comment}


\subsection{Robotic Manipulator}\label{manipulator}
The HVOF coating requirements and environment constraints demand a mid-sized
robotic manipulator (Sec.~\ref{hvof}). A survey was conducted to determine the
most adaptable off-the-shelf manipulator for the application. Overall
simu\-lations and analysis were performed using the OpenRave
\cite{diankov2008openrave}, an environment for simulating motion planning
algorithms for robotics. There are several tools for dynamics simulation of
robots: Gazebo, V-Rep, Webots, and others%
% \cite{ivaldi2014tools}
. OpenRave was selected because of its integral design for real-time control and
execution monitoring, the core functionality for kinematics operations and
physics simulations.

The simulations were performed to analyze the manipulator's work envelope in
the turbine, the required positions for full blade coating, the
manipulator's efforts (torque estimation), and to investigate possible
collisions with mechanical parts. The simulations steps are runner's
blade discretization, manipulator's base computations for full cover,
and robotic manipulator's kinematics and dynamics. 

The blade's surface discretization is the uniform sampling of the blade. A
bounding box of the blade is taken and its surface is uniformly sampled.
The blade samples are the intersections of the blade and rays originating from
each point of the box going inward. The normal vectors of the blade's surface
from each of these intersection points will be the end-effector directions for
coating.
%As an uniform sampling of the
%bounding box does not mean an uniform
% sampling of the blade, the box is oversampled, the blade's samples taken, and
% a k-d tree was generated to remove more than 10~mm close samples (filtering).
%The resulting samples are translated 230~mm in respect with their normal
%vectors, and collisions with the environment are checked.

The manipulator's base computations are to uniformly sample the turbine's
confined space and to calculate the minimum required positions to process all
the blade's samples, considering angle and distance tolerances of the process.
It is a brute force search: for each position, inverse kinematics are computed
to determine the manipulator's joint parameters that provide the desired
positions and orientations of the end-effector.

The kinematic approach described above is not enough to ensure that the robot
will reach the samples. Maximum accelerations and torques should be
investigated. To do this, we employ the well know relation:
$\tau = M(q)\ddot{q} + C(q,\dot{q})\dot{q} + G(q)$
\cite{sciavicco2000differential}, where $\tau$ is the joints' torques, $M$ is
the matrix of moments of inertia, $q$ is the joints' angles, $C$ is the Coriolis
matrix, and $G$ is the gravity vector.
The angular accelerations are derived by differential kinematics:
$\ddot{q}=J^\dagger(\ddot{x}-\dot{q}^TH\dot{q})$, where $H$ is the Hessian
matrix \cite{hourtash2005kinematic}, $\ddot{x}$ is the linear accelerations, and
$J^\dagger$ is the Jacobian matrix pseudoinverse. Therefore, torques can be
analytically estimated by the inverse dynamics in OpenRave and compared to robot
specifications.

% The typical path of HVOF coating is a zigzagging trajectory,
% i.e., it is a deceleration and acceleration process with direction changes. As
% the \textit{ex situ} solution uses a large-sized robotic manipulator, the
% end-effector's direction changes occur outside the blade's range, complying the
% speed requirement in the blade's range. However, while deceleration or
% acceleration, coating material is wasted to the environment or, most commonly,
% to a shadow plate.
% 
% The mid-sized robot for \textit{in situ} operation makes this strategy
% impossible, as, at some base positions, the robot will always be in the blade's
% range. The proposed solution modifies the original circuit of the gases that
% carry the coating particles to a new 2-way controlled circuit, in which there is
% a directional valve that changes the flow path from the thermal spray gun to return to tank
% circuit. The directional valve is autonomously actuated accordingly to the robotic
% manipulator's trajectory, i.e., the valve will ``shut'' while end-effector
% changes its direction. Therefore, the solution provides coating material
% savings, as the non-used particles are stored for future usage, and the hard
% coating speed requirement is met.

% \begin{figure}[h!]
%    \centering
%    \includegraphics[width=0.9\columnwidth]{figs/mecanica/Circuito_HVOF_mod_en.PNG}
%    \caption{Simplified view of HVOF circuit with shutter}
%    \label{fig::circuito_hvof}
% \end{figure}

 

\subsection{Mechanical system}
% 0.75 pages

EMMA's mechanical system is non-actuated rails for manipulator's base
transportation, positioning and fixation. It comprises two rails, forming two
prismatic joints (P-P). The first rail is parallel to the
turbine axis, and it is responsible for the transportation of the manipulator from
hatch to close to the blade. The secondary rail is assembled from the first by
a rotational joint (R), which allows to position the upper rail parallel to the
blade's surface.

\begin{figure}
	\centering
	\includegraphics[width=1.0\columnwidth]{figs/mecanica/EMMA_Base_Secundaria_03.PNG}
    \caption{Customized base: primary and secondary rails.}
    \label{fig:base}
\end{figure}

The hatch limits the size of the rail in terms of weight
and geometry, thus a modular concept was adopted, such that the small modular
parts can be easily and manually assembled inside the turbine. Each module
contains all the necessary components to support, transport and position the
manipulator along it. Thus, the modules can be simply assembled in
sequence to increase the overall length of the prismatic joint. The robot and
the modular parts of the mechanical system are brought to the confined
space of the turbine by a hoist.

A two parallel profiled rail system with a four carriage
setup\footnote{Profile Rail Guides LLT: Mounting maintenance and repair
instructions, SKF Group.} was adopted. %\cite{SKF_2013}.
This configuration creates a balanced force couple, eliminating the reaction
moments in each carriage. The frame structure is formed by aluminum
profile, since it is lightweight, corrosion resistant, geometrically flexible,
and modular (easy rail reconfiguration by changing few parts or adding anchor
arms).

The resulting frame structure is slender and lightweight, demanding careful
dynamic analysis of its integrity and stiffness. A Finite Element Analysis (FEA)
was performed to evaluate the stresses, strains and forces along the structure.
FEA is also used to specify the frame components, such as the profile's size,
and anchors quantity, dimensions, directions, and attachment points.

Since the draft tube and runner area are curved and sloped, properly fixing
the mechanical base is a major challenge in EMMA. The draft tube is composed of a
ferromagnetic steel material, hence magnetic fixtures are solution for base
attachment without environment modifications, as welding.

\begin{comment}
Common magnetic fixture products are temporary magnetic equipments used to hoist
and transport materials in an industrial plant. In EMMA, it is used to attach
and anchor the rail's modules to the ground.
\end{comment}

\subsection{Calibration}
 
The calibration process was divided in two different approaches depending on the
element to be localized and its caracteristics: the pose estimation of the
robot; and the blade.
   
The attachment of markers on the manipulator can be performed with high
precision and repeatability, thus reflective spheres were chosen as reference
points in the process of robot localization. These spheres are identified inside
the point cloud, the scene provided by the 3D laser sensor, using the 3D Hough
Transform method \cite{camurri20143d}. The 3D Hough Transform, as the 2D version
of the method, is the search of an object in its (discretized) parameter space.
A sphere has four parameters: three for the position of its center, and one for its radius.
For each point on the point cloud, it is assigned a collection of voxels on the
discretized parameter space, corresponding to the possible spheres passing
through that point. As in a voting process, the voxels with the greatest number
of points assigned to define the parameters of the most probable spheres. There
might be computational issues depending on the size of the parameter space, but
it can be mitigated by exploiting previous knowledge regarding the expected
radius and viable region for the robot inside the enviroment. Further
improvements can be achieved through the use of the normal vector at each point,
taking into accoun the fact that the center of the sphere is in the direction of
the normal vector, thus reducing the number of assigned voxels per point.

Fixing any marker on the blade would require its own calibration to ensure a
consistent reference point, thus the pose estimation of the blade must rely only
on the intrinsec properties of its surface geometry. The information must be
extracted from the point cloud and compared to a reference model previously
stored. The characteristics of the point clouds are represented by local
descriptors, i.e., each interest point on the blade is associated with a piece
of information about its local neighborhood. Given the sets of features from the
model and the scence,  it is possible to determine the correspondence between
their descriptors. If enough correspondences are found in the scene, above a
threshold, the blade is identified and the position can be determined
\cite{Tombari2010a}.

However, as the blade is a large and smooth surface, the neighborhood of each
point may introduce similiar information, creating ambiguous descriptors that
degrade the matching perfomance, thus it is fundamental to wisely pick the
interest points.
The ambiguous descriptors provide information about perpendicular translation to
the supporting plane, and the two rotations associated with it, thus no
information about the other DOFs, making the alignment to ``slide'' from the
correct transformation. Therefore, the interest points were sampled to diversify
the normal vectors' directions \cite{Rusinkiewicz2001}, reducing the number of
samples to have a descriptor associated, when compared to a uniform sampling,
lowering the computational cost. Once the descriptors were estimated, the
correspondences are determined if the euclidean distance between a descriptor in
the scene and in the model is lower than a threshold. Each correspondence vote
for a specific pose and scale factor in the Hough space. After an instance of
the model is found, it is performed the Iterative Closest Point (ICP) matching
with the full resolution point clouds to realize a fine alignment and compensate
any discrepancy introduced by the sampling. 

Finally, after the blade's and the
manipulator's identification in respect to a common coordinate system, e.g., the
origin of the laser sensor, it is possible to determine the transformation marix
between them, which is the input to the trajectory and coverage algorithms. It
is important to note that in every step of the operation, in which either the
manipulator's base or the blade are moved, the system must be recalibrated.



