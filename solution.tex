\section{Solution}

%2.5 pages
%TODO Renan: Solucao-Intro
%TODO Estevão: Solucao-Rails
%TODO Estevão: Solucao-Magnetic Base
%TODO Estevão: Solucao-Shutter
%TODO Renan: Solucao-Manipulator
%TODO Renan: Solucao-Software
%TODO Gabriel: Solucao-Software

Regarding the environment constraints detailed in Sec.~\ref{problem}, the
conceptual solution of EMMA is a mid-sized robotic manipulator with a modular
rail typed base. As the manipulator can not fully cover the blade surface in a
fixed position, and the robot locomotion is complex in the turbine's
environment, a customized rail should provide extra degrees of freedom (DOF) to
the system.

Overall simulations and analysis were performed using the Open Robotics
Automation Virtual Environment (OpenRave) for blade discretization, coating
strategy, coating segmentation (partitioning) and base positions, manipulator
kinematics and dynamics, and motion planning, in real-world
\cite{diankov2008openrave}. 

\begin{comment}
Several concepts for robotic base were
simulated and the simpler solution in constructive terms was chosen,
consisting of a Prismatic-Rotational-Prismatic-Prismatic (PRPP) base.
%TODO Estevâo: breve justificativa. ``Simpler'' ficou vago. Modular?



The coating solution used in industry today is to oversize the robot
in means of its reach, such that one fixed and large base is enough to do all 
the process without moving the robot's base. 
This is feasible because there are no constraints regarding to its size and no
other objects but the blade in an optimal position, in a specially designed room.

Because of the small access to the interior of the turbine and the narrow space
between the blades, this solution requires a midsize to small robot. 
These manipulators cannot reach all the surface of the blade without moving its
base, so an external base need to provide degrees of freedom (DOF) that allow
the manipulator to reach all the surface.
A geometric and kinematic study was made to find the required positions of the
robot's base that allow the complete coverage of the blade.
From this, were studied several base concepts regarding to the DOF and joints of
the base. The simpler solution in constructive terms was chosen, and consists of
a Prismatic-Rotational-Prismatic-Prismatic (PRPP) joints.
\end{comment}


\subsection{Robotic Manipulator}

\subsection{Base}
% 0.75 pages

EMMA's base comprises two rails, forming two prismatic joints. The
first, or primary rail, is parallel to the turbine axis, and it is responsible
for the transportation of the robotic manipulator between the hatch and the
blade. The secondary rail is is parallel to the blade's surface, and it is
assembled from the first rail by a rotational joint for angle adjustment. A
third prismatic joint is placed in the secondary rail for height adjustment of
the robotic manipulator, since it can not reach all blade. These rails guarantee
the full coating cover, as they allow the robotic manipulator movement through
the constrained space, in a 4-DOF base, Prismatic-Rotational-Prismatic-Prismatic
(PRPP) joints.

The hatch limits the size of the rail in terms of weight and geometry, thus a
modular concept was design, such that the small modular parts can be easily and manually assembled inside the
turbine. The base is a modular two parallel profiled rail system %TODO Estevao:
% referencia
with a four carriage setup. In this configuration, the reaction moments of the
base are cancelled by a force couple provided by each couple of carriages. The
% and the loads are divided in ore components.
%TODO Estevao: colocar figura
comercial modules are aluminum structures for corrosion resistance,
lightweight, geometric flexibility, and modularity, as it is possible to
increase/decrease the rail length/width/height by changing few parts or adding
anchor arms. %TODO Estevao: referencia para as propriedades do aluminio
% (aplicacao)

\begin{comment}
A comercial modular profile system in aluminum is used to form the main
structure in wich the rails are mounted. The main advantages of this kind of
system are a corrosion resistant and light structure with a large flexibility
in terms of geometry and modularity so that it is possible to increase or
decrease length, width or height of the structure changing few parts, add
anchor arms and many other accessories quickly.
\end{comment}

% Because of the constraints to access the turbine, the structure of the base
% can't be oversized in terms of weight and geometry. So 
The resulting base is a slender and lightweight frame, such that a careful
analysis is necessary to evaluate the structural integrity and rigidity, in
respect to the dynamic loads of the robotic manipulator. A Finite Element
Analysis (FEA) was performed to study the stresses, strains and resultant forces
along the structure. FEA is also used to size the frame components, such as
the profile's size, and to size the anchors, its directions, and attachment to
provide the greater base's rigidity. The rigidity is a major concern, it is
required for the hard coating process, since a non-rigid base propagates
vibrations to the robotic manipulator's end-effector with high amplitudes, which
may compromise coating quality.
%TODO Estevao: software do FEA


\subsection{Magnetic bases}
%0.25 page

The draft tube and runner area are conical shape structures, curved and
sloped. The environment modification, as wielding and drilling, should be
avoided, thus properly fixing the base on the ground is a major challenge in
EMMA project. The draft tube is composed of a ferromagnetic steel material,
hence magnetic fixtures are a suitable solution for base attachment. 

Comercials magnetic bases are non-permanent magnetic equipment used in industry
to hoist and transport materials in a plant.
For this solution it is used to attach and anchor the structure of the base to
the ground. 
Tests were conducted in a sample of surfaces inside the turbine in order to
evaluate the load capacity of this equipment due to the different shapes,
surface finishing and curvatures that are found there.
The results concluded that this equipment is well suited for this application.


\subsection{Shutter}

The HVOF process requires a $40~m/min$ speed along the blade's surface, for a
best coating quality. 
The current industrial process uses an oversized robot that is able to
cover the blade from side to side. In this manner, the changes in direction, and
consequently changes in speed, occurs outside the blade's coating area.
This permits the robot to achieve the required speed again, when back to the
front of the surface.
While the HVOF gun is in the outside region of the blade, changing its
direction, it wastes the material to the environment or, most commonly to a
shadow plate.

The size of the robot for the \textit{in situ} solution makes this strategy
impossible, because the robot will be inevitably in an intern region of the blade and will
needs to change the direction of the HVOF gun several times in these
positions. 

Thus, the solution is basically to ''shut'' the process while the arm changes
its trajectory and open again when the required speed is achieved.
The shutter is a system that modifies the original circuit of the gases that
carries the coating particles to a new 2-way controlled circuit in wich there 
is a directional valve that changes the flow path from the thermal spray gun to
a return to tank circuit. 
The directional valve control module have informations about the pre-programmed
trajectory of the manipulator so that it is autonomously actuated.
In this way, there is much less waste of material, whereas the non-used
particles can be utilized again, and the hardcoating requirements can be met.


\subsection{Software}