\section{Solution}

%2.5 pages
%TODO Renan: Solucao-Intro
%TODO Estevão: Solucao-Rails
%TODO Estevão: Solucao-Magnetic Base
%TODO Estevão: Solucao-Shutter
%TODO Renan: Solucao-Manipulator
%TODO Renan: Solucao-Software
%TODO Gabriel: Solucao-Software

Regarding the environment constraints detailed in Sec.~\ref{problem}, the
conceptual solution of EMMA is a mid-sized robotic manipulator with a modular
rail typed base. As the manipulator can not fully cover the blade surface in a
fixed position, and the robot locomotion is complex in the turbine's
environment, a customized rail should provide extra degrees of freedom (DOF) to
the system.

Overall simulations and analysis were performed using the Open Robotics
Automation Virtual Environment (OpenRave) for blade discretization, coating
strategy, coating segmentation (partitioning) and base positions, manipulator
kinematics and dynamics, and motion planning, in real-world
\cite{diankov2008openrave}. 

%TODO Estevâo: inserir aqui a introduçao das subseçoes. O que vai ser dito
% adiante? Não em tópicos, mas dentro do parágrafo anterior.

\begin{comment}
Several concepts for robotic base were
simulated and the simpler solution in constructive terms was chosen,
consisting of a Prismatic-Rotational-Prismatic-Prismatic (PRPP) base.
%TODO Estevâo: breve justificativa. ``Simpler'' ficou vago. Modular?



The coating solution used in industry today is to oversize the robot
in means of its reach, such that one fixed and large base is enough to do all 
the process without moving the robot's base. 
This is feasible because there are no constraints regarding to its size and no
other objects but the blade in an optimal position, in a specially designed room.

Because of the small access to the interior of the turbine and the narrow space
between the blades, this solution requires a midsize to small robot. 
These manipulators cannot reach all the surface of the blade without moving its
base, so an external base need to provide degrees of freedom (DOF) that allow
the manipulator to reach all the surface.
A geometric and kinematic study was made to find the required positions of the
robot's base that allow the complete coverage of the blade.
From this, were studied several base concepts regarding to the DOF and joints of
the base. The simpler solution in constructive terms was chosen, and consists of
a Prismatic-Rotational-Prismatic-Prismatic (PRPP) joints.
\end{comment}


\subsection{Robotic Manipulator}

\subsection{Base}
% 0.75 pages

EMMA's base comprises two rails, forming two prismatic joints. The
first, or primary rail, is parallel to the turbine axis, and it is responsible
for the transportation of the robotic manipulator between the hatch and the
blade. The secondary rail is is parallel to the blade's surface, and it is
assembled from the first rail by a rotational joint for angle adjustment. A
third prismatic joint is placed in the secondary rail for height adjustment of
the robotic manipulator, since it can not reach all blade. These rails guarantee
the full coating cover, as they allow the robotic manipulator movement through
the constrained space, in a 4-DOF base, Prismatic-Rotational-Prismatic-Prismatic
(PRPP) joints.

The hatch limits the size of the rail in terms of weight and geometry, thus a
modular concept was design, such that the small modular parts can be easily and manually assembled inside the
turbine. The base is a modular two parallel profiled rail system %TODO Estevao:
% referencia
with a four carriage setup. In this configuration, the reaction moments of the
base are cancelled by a force couple provided by each couple of carriages. The
% and the loads are divided in ore components.
%TODO Estevao: colocar figura
comercial modules are aluminum structures for corrosion resistance,
lightweight, geometric flexibility, and modularity, as it is possible to
increase/decrease the rail length/width/height by changing few parts or adding
anchor arms. %TODO Estevao: referencia para as propriedades do aluminio
% (aplicacao)

\begin{comment}
A comercial modular profile system in aluminum is used to form the main
structure in wich the rails are mounted. The main advantages of this kind of
system are a corrosion resistant and light structure with a large flexibility
in terms of geometry and modularity so that it is possible to increase or
decrease length, width or height of the structure changing few parts, add
anchor arms and many other accessories quickly.
\end{comment}

% Because of the constraints to access the turbine, the structure of the base
% can't be oversized in terms of weight and geometry. So 
The resulting base is a slender and lightweight frame, such that a careful
analysis is necessary to evaluate the structural integrity and rigidity, in
respect to the dynamic loads of the robotic manipulator. A Finite Element
Analysis (FEA) was performed to study the stresses, strains and resultant forces
along the structure. FEA is also used to size the frame components, such as
the profile's size, and to size the anchors, its directions, and attachment to
provide the greater base's rigidity. The rigidity is a major concern, it is
required for the hard coating process, since a non-rigid base propagates
vibrations to the robotic manipulator's end-effector with high amplitudes, which
may compromise coating quality.
%TODO Estevao: software do FEA


\subsection{Magnetic bases}
%0.25 page

The draft tube and runner area are conical shape structures, curved and
sloped. The environment modification, as wielding and drilling, should be
avoided, thus properly fixing the base on the ground is a major challenge in
EMMA project. The draft tube is composed of a ferromagnetic steel material,
hence magnetic fixtures may be suitable for base attachment. 

Common magnetic fixture products are temporary magnetic equipments used to hoist
and transport materials in an industrial plant. In EMMA, it is used to attach
and anchor the rail's modules to the ground. Field tests were conducted in the
draft tube for payload evaluation, at different equipment's orientations. The
results concluded that the magnetic fixture is sufficient for this application.
%TODO Estevao: resumir o teste que foi feito em uma frase dentro deste
% parágrafo. Colocar qual o dispositivo selecionado.


\subsection{Shutter}

To achieve a good coating quality, the HVOF process requires that the robotic
manipulator's end-effector moves with a constant $40~m/min$ speed along the
blade's surface. The typical path of HVOF coating is a zigzagging trajectory,
i.e., it is a deceleration/acceleration process with direction changes. The
\textit{ex situ} solution uses a large-sized robotic manipulator, which can
fully cover the blade in a fixed position, and the end-effector's direction
changes occur outside the blade's range, complying the speed in the blade's
range. However, while deceleration/acceleration, coating material is wasted to
the environment or, most commonly to a shadow plate.

The mid-sized robot for \textit{in situ} operation makes this strategy
impossible, as, at some base positions, the robot will always be in the blade's
range, and changing the end-effector's direction will be necessary. The proposed
solution modifies the original circuit of the gases that carry the coating
particles to a new 2-way controlled circuit, in wich there is a directional
valve that changes the flow path from the thermal spray gun to return to tank
circuit. %TODO Estevao: figura do processo
The directional valve is autonomously actuated accordingly to the robotic
manipulator's trajectory, i.e., the valve will ``shut'' while end-effector's
direction changes. Therefore, the solution provides coating material savings,
as the non-used particles are stored for future usage, and the hard coating
speed requirement is met.


\subsection{Software}