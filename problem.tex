\section{The hard coating process}\label{hvof}

Hydropower runner's blades are typically eroded by cavitation and abrasion
phenomena, resulting in hydraulic profile deformation, thus efficiency
reduction. The High Velocity Oxygen Fuel (HVOF) coating is a preventive
solution for erosion, and creates a lamellar structure. 

The HVOF is a 2000~hp power process which consists of
spraying coating particles by an 8~kg spray gun, through a flame with mixed
gases. To achieve the best coating layer, the spray gun should be at a fixed
210~mm to 240~mm distance, and $90^\circ \pm 30^\circ$ angle, in respect to the
metallic surface plane of the blade; and the gun should move at 40~m/min speed
along the path \cite{li2002effect}.  Besides, for a regular coating cover, the
trajectory is 3~mm spaced horizontal lines crossing the blade's surface (coating step), which requires great positional accuracy of the robot.
The common solution which meets the requirements for \textit{ex situ} HVOF
coating is a robotic manipulator with a blade-sized workspace in a fixed position.

\section{The problem}\label{problem}

The problem is to design a robotic system for \textit{in situ}
hydropower runner's blades. Accessibility is a major problem: the robot must be
brought to the turbine through a hatch; and it must operate in the confined,
curved, slippery and harsh environment of the turbine. Besides, there are
several control and calibration problems, as trajectory planning, and robot localization.

The mechanical challenges are robot locomotion, base stiffness, and
fixation. The robot should be transported and positioned in turbine's
environment, as the access is generally far from the runner's blade.
The stiffness is required for the hard coating process, since a base's
vibrations are propagated to the manipulator's end-effector with high
amplitudes, compromising the coating quality. 

Regarding calibration, the relative position between the manipulator and the
blade is not fixed. The system calibration consists in the identification of
the manipulator and blade, and their pose estimation in respect to the turbine
interior. Due to the environment's light conditions, 3D laser sensing
technology should be used to map the topography of the blade, and the robotic
system. 

Large-sized manipulators are not suitable for \textit{in situ} operations, due
to the accessibility, and conventional compact manipulators do not
have the required work envelope or payload for the task. Therefore,
customized or mid-sized manipulators should be investigated by kinematics and
dynamics simulations. Besides, the modeling of curvilinear space,
the automatically trajectory generation, and the robot position and velocity
control comprise the robot control strategy.


\begin{comment}

A bulb type turbine has the following points of interest for solution
development: 1) the variable pitch propeller, or Kaplan \textbf{blades}; 2) the
variable pitch guide vanes, or \textbf{wicket gates}; 3) the \textbf{runner
area}; 4) the \textbf{draft tube}; and 5) an access for regular
maintenance, or \textbf{hatch}. The Jirau's turbine is the case study of EMMA,
thus a 3D CAD model was built with SolidWorks\raisebox{1ex}{\textregistered}
for simulation and solution analysis (Fig.~\ref{fig::ambiente3d}).

\begin{figure}[h!]
\centering
	\includegraphics[width=\columnwidth]{figs/problem/ambiente_3d.PNG} 
	\caption{Jirau's hydropower turbine in a 3D CAD model.}
	\label{fig::ambiente3d}
\end{figure}

\end{comment}