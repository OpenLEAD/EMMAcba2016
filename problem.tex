\section{The problem}

%0.5 page

%TODO Renan: Problem

Hydropower runner's blades are typically eroded by cavitation and abrasion
phenomena, resulting in hydraulic profile deformation, thus efficiency
reduction. The High Velocity Oxygen Fuel (HVOF) coating is a preventive
solution for erosion, and creates a lamellar structure, increasing
power generation efficiency. 

The runner's blade HVOF process consists of spraying coating particles
by an 8~kg spray gun. To achieve the best coating layer, the spray gun should be at a
fixed 230~mm to 240~mm distance and $90^o \pm 30^o$ angle, in respect
to the metallic surface plane of the blade; the gun should move with 40 m/s
speed along the blade; and the precision (or coating step) should be 3~mm for a
regular full blade cover, due to the 5~mm diameter of the gun's output (flame).

Regarding the requirements, the common solution for \textit{ex situ} HVOF
coating is a robotic manipulator with a blade-sized workspace in a fixed
position. However, \textit{in situ} runner's blade HVOF coating represents
great technological challenges. The robotic system must overcome
turbine's environmental aspects, such as, the constrained space, the slippery
and sloping floor, and the unfriendly atmospheric conditions. Therefore, a
large-sized industrial robotic manipulator is not suitable for the operation.

As the runner can be manually rotated, the main problem and objective of the
proposed robotic system is to hard coat the two sides of one blade. The
robotic system must comply with the HVOF requirements, it must overcome the
constraints and logistical challenges of the environment, and operate
autonomously. 

\begin{comment}
In the case of the Jirau hydroelectric dam, the coating of turbine's blades is
performed before turbine assembling and installation. However, the abrasion due
to a large number of particles and sediment in the Madeira river and the recent
identified cavitation require recoating in short intervals \citep{santa2009slurry}.
\end{comment}