\section{Conclusion and future work}
% Renan
% 0.5 pages

In this paper, we presented the concept solution and detailed some
aspects of the EMMA robotic system, for the problem of \textit{in situ} hard
coating system of hydropower turbine. All the robotic systems seen so far were
developed for repairing by wielding or inspection applications, but the
challenging requirements and constraints of the \textit{in situ} HVOF coating
procedure were not yet considered.

Climbers and robotic systems on rails attached to the blade were investigated as
general solutions. However, these systems typically have small payloads and
low speed, which do not meet the process' requirements. The design of these systems
would be very complex, due to the blade's temperature variation, and the
locomotion and adhesion mechanisms. The proposed system is simpler, similar to
the \textit{ex situ} solution, and the modular concept offers flexibility to
logistics.

EMMA methodology and some elements of the system was presented: the customized
modular rails; manipulator workspace analysis, kinematics, and dynamics; and
the system localization and calibration. Field tests and simulations were
performed and preliminary results show the feasibility of the concept solution.

Ongoing implementations and future work include: Software integration,
trajectory generation,and control strategy; mechanical
vibration analysis; and field tests with the designed robotic system.