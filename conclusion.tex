\section{Conclusion and future work}
% Renan
% 0.5 pages

In this paper, we presented the concept solution and general overview of the
EMMA project, which endeavors to develop a hard coating system for \textit{in
situ} hydropower turbine. All the robotic systems seen so far were developed for
repairing by wielding or inspection applications, but the challenging
requirements and constraints of the \textit{in situ} HVOF coating procedure
were not yet considered. 

Climbers and robotic systems on rails attached to the blade were investigated as
general solutions. However, these systems typically have low payloads and low
speed, which do not meet the process' requirements. The design of these systems
would be very complex, due to the blade's temperature variation, and the
locomotion and adhesion mechanisms. The proposed system is simpler, similar to
the \textit{ex situ} solution, and the modular concept offers flexibility to
logistics.

The methodology of the development of EMMA project and some steps
of the autonomous system was presented: the customized rails and mechanics;
robotic manipulator, kinematics, dynamics and simulations; and the system
localization and calibration. Field tests and real-world simulations were
performed and preliminary results show the feasibility of the concept solution.

Ongoing implementations and future work include:
\begin{enumerate}
 	\item \textit{Control}: control and path planning of the robotic
 	manipulator;
  	\item \textit{Autonomous operation}: software integration of calibration,
  	localization, control and path planning;
  	\item \textit{Field tests}: field test with the designed robotic system;
\end{enumerate}